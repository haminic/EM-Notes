\chapter{พลศาสตร์ไฟฟ้า}
\section{แรงเคลื่อนไฟฟ้า}
\subsection{กฎของ Ohm}

ในการเคลื่อนย้ายประจุให้เกิดกระแสก็จะต้องออกแรง เราจึงมาหาความสัมพันธ์ระหว่างแรงกับกระแสกันก่อน

พิจารณาสายไฟที่มีอิเล็กตรอนอิสระอยู่ $n$ อนุภาคต่อหน่วยปริมาตรและแต่ละอิเล็กตรอนมีมวล $m_e$ ประจุ $e$ และสมมติมีสนามแรง $\vb{f}$ (ต่อหน่วยประจุ) กระทำอยู่กับทั้งสาย แรง $\vb{f}$ จะทำให้อิเล็กตรอนเคลื่อนที่ด้วยอัตราเร่ง $a$ ก่อนที่จะชนกับอิเล็กตรอนอีกอนุภาคจนทำให้อัตราเร็ว (โดยเฉลี่ยทั้งหมดแล้ว) กลับมาเป็น $0$ อีกครั้ง โดยถ้าสมมติว่าอัตราเร็วของอิเล็กตรอนเนื่องจากความร้อนเท่ากับ $v_\txt{thermal}$ และมีระยะทางเฉลี่ย $\lambda$ ระหว่างการชน เนื่องจาก $v_\txt{thermal}$ มีค่าสูงมาก จึงประมาณได้ว่าความเร่งที่เกิดขึ้นนั้นมีผลน้อยมาก จึงได้เวลาโดยเฉลี่ยก่อนที่จะชนกับอิเล็กตรอนอีกอนุภาคคือ
\[ 
t = \frac{\lambda}{v_\txt{thermal}} 
\]
ก็จะได้ขนาดของความเร็วเร็วเฉลี่ยหรือ\emph{อัตราเร็วลอยเลื่อน} (\emph{drift velocity}) เท่ากับ
\begin{equation}
    v_d = \frac{1}{2}at = \frac{a\lambda}{2v_\txt{thermal}}\label{drift}
\end{equation}
ดังนั้นกระแสจึงเท่ากับ
\begin{equation} 
    \vb{J} = ne\vb{v}_d = ne\frac{\lambda\vb{a}}{2v_\txt{thermal}} = \ab(\frac{\cancel{n}e\lambda}{2v_\txt{thermal}\cancel{n}m_e})\vb{F} = \ab(\frac{ne^2\lambda}{2v_\txt{thermal}m_e})\vb{f}\label{ohmapprox} 
\end{equation}
จะเห็นว่าโดยปกติแล้วสำหรับวัสดุทั่วไป $\vb{J}$ จึงแปรผันตรงกับ $\vb{f}$:
\begin{eqbox}{สมการการแปรผันตรงของกระแสกับแรง} 
    \vb{J} = \sigma\vb{f} 
\end{eqbox}
โดยที่ $\sigma$ เป็นค่าคงที่ที่เรียกว่า\emph{สภาพนำไฟฟ้า} (\emph{conductivity}) ของสสารนั้น (ถ้าสสารเป็นตัวนำในอุดมคติก็จะมี $\sigma = \infty$) และ $\rho\equiv 1/\sigma$ เรียกว่า\emph{สภาพต้านทาน} (\emph{resistivity}) โดยถ้าแรงที่ใช้เป็นแรงทางไฟฟ้า\underline{เท่านั้น}โดยมีส่วนของแรงแม่เหล็กน้อยมาก ๆ ก็จะได้
\begin{ieqbox}{กฎของ Ohm} 
    \vb{J} = \sigma\vb{E}\label{ohme} 
\end{ieqbox}
และจาก (\ref{drift}) จะได้ว่า
\begin{eqbox}{อัตราเร็วลอยเลื่อน}
    v_d = \frac{a\lambda}{2v_\txt{thermal}} = \frac{eE}{2m_ev_\txt{thermal}}\lambda = \frac{eE}{2m_e}\tau' = \frac{eE}{m_e}\tau\label{drifte}
\end{eqbox}
เมื่อ $\tau'$ คือเวลาเฉลี่ยระหว่างการชนสองครั้งที่ติดกันและ $\tau$ คือเวลาเฉลี่ยหลังการชนครั้งก่อนหน้า (โดยใช้เวลาเฉลี่ยบนการสุ่มเลือกอิเล็กตรอน)

หมายเหตุ: \emph{สมการ (\ref{ohmapprox}) และ (\ref{drifte}) เป็นเพียงการประมาณหยาบ ๆ แบบกลศาสตร์ดั้งเดิมเท่านั้น จึงไม่สามารถนำมาใช้หา $\sigma$ และ $v_d$ ได้ในสสารจริง ๆ และยิ่งไปกว่านั้น ในความเป็นจริงแล้วยังมีวัสดุบางชนิดที่ไม่เป็นไปตามกฎการแปรผันตรงนี้อีกด้วย เราจะเรียกวัสดุที่เป็นไปตามกฎของ Ohm ว่าเป็นวัสดุ Ohmic}

สังเกตว่าในการทำให้ความต่างศักย์มากขึ้น $k$ เท่าระหว่างขั้ว\emph{อิเล็กโทรด} เราจะต้องเพิ่ม $Q$ ไป $k$ เท่า ทำให้ $\vb{E}$ เพิ่ม $k$ เท่าและจาก (\ref{ohme}) จะได้ว่า $\vb{J}$ และ $I$ ก็เพิ่ม $k$ เท่าเช่นกัน ก็จะได้กฎของ Ohm ในอีกรูปแบบ:
\begin{ieqbox}{กฎของ Ohm ในรูปกระแสและความต่างศักย์} 
    V = IR \label{ohmv}
\end{ieqbox}
เมื่อ $R$ เป็นค่าคงที่\emph{ความต้านทาน}ระหว่างสองจุดนั้น (ในการคำนวณหาความต้านทานใช้ (\ref{ohme}) ตามในแต่ละระบบได้เลย) โดย $R$ นี้มีหน่วย SI คือ $\unit{\ohm}$ (ohm)

ในกรณีที่กระแสไหลแบบคงที่ในสสารเนื้อเดียวกันที่เป็นไปตามกฎของ Ohm จาก (\ref{gdjzero}) จะได้ว่า
\begin{equation} 
    \gd\cdot\vb{E} = \frac{1}{\sigma}\gd\cdot\vb{J} = 0 
\end{equation}
ดังนั้นในบริเวณที่สสารเป็นไปตามกฎของ Ohm ก็จะไม่มีประจุตกค้างอยู่ภายในเลย จึงทำให้สามารถใช้ทริคในการแก้ศักย์และสนามจากสมการ Laplace ได้ตามปกติ

สุดท้าย จาก (\ref{ohmapprox}) เนื่องจากแรงที่ออกนั้นไม่ส่งผลในอัตราเร็วลอยเลื่อนเพิ่มขึ้นเลย ดังนั้นพลังงานส่วนมากจากการชนจะถูกเปลี่ยนเป็นความร้อน โดยถ้ามีประจุไหลต่อเวลาเท่ากับ $I$ โดยศักย์ของประจุลดลง $V$ ก็จะได้
\begin{ieqbox}{กฎการให้ความร้อนของ Joule} 
    P = IV = I^2R = \frac{V^2}{R} 
\end{ieqbox}

\subsection{แรงเคลื่อนไฟฟ้า}

โดยปกติแล้วในวงจรไฟฟ้าจะมีแรงสองแรงในการทำให้ประจุเคลื่อนที่คือแรงจากแหล่งกำเนิด ($\vb{f}_s$) ซึ่งโดยปกติแล้วแรงนี้จะอยู่แค่ในบริเวณแหล่งกำเนิดเท่านั้น และอีกแรงคือแรงจากสนามไฟฟ้าที่จะเป็นตัวที่ช่วยทำให้กระแสไหลด้วย $I$ คงที่ตลอดทั้งสาย ดังนั้นแรงต่อประจุโดยรวมจะเท่ากับ
\[ 
\vb{f} = \vb{f}_s + \vb{E} 
\]
แต่แรง $\vb{E}$ ที่ช่วยให้กระแสไหลคงที่มาจากไหนล่ะ? เราลองพิจารณาทีละขั้นตอน ดังนี้:
\begin{enumerate} 
    \item เมื่อเริ่มต่อสายไฟกับแบตเตอรี่ จะเกิดแรง $\vb{f}_s$ ทำให้เกิดกระแสไหลออก โดยถ้ากระแสในสายไฟเปล่านี้เริ่มไหลไม่คงที่ จะทำให้มีประจุสะสมเกิดขึ้นจึงมี $\vb{E}$ ต้านกระแสส่วนที่เร็วเกินไปและเสริมในส่วนที่ช้าเกินไป
    \item ที่บริเวณตัวต้านทานก็เช่นเดียวกัน จะต้องมีกระแสเท่ากับนอกตัวต้านทาน แต่คราวนี้ประจุจะสะสมไปเรื่อย ๆ จนกว่าสนามไฟฟ้าที่เกิดขึ้นจะมากพอที่จะพลักประจุผ่านตัวต้านทานไปได้ด้วยกระแสเท่ากับข้างนอก (ตาม (\ref{ohme})) โดยกระเกิดประจุสะสมที่ฝั่งหนึ่งของตัวต้านทานก็จะทำให้เกิดประจุสะสมที่ขั้วของแบตเตอรี่ด้วย
    \item อีกขั้วของแบตเตอรี่ก็จะเกิดกระบวนการเช่นเดียวกับ 1. และ 2. แต่ในทิศและขั้วตรงข้าม
\end{enumerate}
เราจึงนิยามผลของแรงทั้งหมดภายในวงจรว่า\emph{แรงเคลื่อนไฟฟ้า}หรือ \emph{emf} (\emph{electromotive force}: $\emf$):
\begin{defbox}{แรงเคลื่อนไฟฟ้า} 
    \begin{equation} 
        \emf \equiv \oint \vb{f}\cdot\odif{\vbs{\ell}} = \oint \vb{f}_s\cdot\odif{\vbs{\ell}} 
    \end{equation} 
\end{defbox}
เนื่องจากสนามไฟฟ้าสถิต $\oint \vb{E}\cdot\odif{\vbs{\ell}} = 0$ โดย $\emf$ นี้มีหน่วยเป็น $\unit{V}$ เช่นเดียวกับศักย์ไฟฟ้า

หมายเหตุ: \emph{emf นี้นิยามเป็นค่า ณ ขณะหนึ่งเท่านั้น ดังนั้นเมื่อสายไฟขยับ เราจะใช้ $\odif{\vbs{\ell}}$ เป็นทิศเดียวกับสายไฟจริง ๆ ไม่ต้องคำนึงถึงความเร็ว}

พิจารณาในสภาวะสมดุลหลังจากต่อแบตเตอรี่: สมมติแหล่งกำเนิดเป็นแบตเตอรี่ไร้ความต้านทาน ($\sigma = \infty$) ก็จะได้ว่าแรงที่ออกในการเคลื่อนประจุเป็น $0$ ดังนั้น $0 = \vb{f} = \vb{f}_s + \vb{E}$ ก็จะได้
\begin{equation} 
    V = -\int_{\vb{a}}^{\vb{b}} \vb{E}\cdot\odif{\vbs{\ell}} = \int_{\vb{a}}^{\vb{b}} \vb{f}_s\cdot\odif{\vbs{\ell}} = \oint \vb{f}_s\cdot\odif{\vbs{\ell}} = \emf 
\end{equation}
แต่ถ้าแบตเตอรี่นี้มีความต้านทาน $r$ (หมายความว่าถ้าตัดแรง $\vb{f}_s$ ออกแล้วความต่างศักย์ $V_\txt{off} = \int\vb{E}_\txt{off}\cdot\odif{\vbs{\ell}} = Ir$) สมการด้านบนจะไม่เป็นจริง โดยจะได้
\begin{equation} 
    V = -\int_{\vb{a}}^{\vb{b}} \vb{E}\cdot\odif{\vbs{\ell}} = \int_{\vb{a}}^{\vb{b}} \ab(\vb{f}_s - \frac{\vb{J}}{\sigma})\cdot\odif{\vbs{\ell}} = \emf + \int_{\vb{a}}^{\vb{b}} \vb{E}_\txt{off}\cdot\odif{\vbs{\ell}} = \emf - V_\txt{off} = \emf - Ir 
\end{equation}

\subsection{แรงเคลื่อนไฟฟ้าจากสายไฟเคลื่อนที่}

เราสามารถเหนี่ยวนำเส้นลวดให้เกิด emf ได้โดยอาศัยสนามแม่เหล็ก ซึ่งเป็นวิธีที่\emph{เครื่องกำเนิดไฟฟ้า} (\emph{generator}) ใช้ในการสร้างกระแสไฟฟ้า โดยยกตัวอย่างเช่น ถ้าเราเอาสายไฟรูปสี่เหลี่ยมมุมฉากที่กว้าง $h$ ไปวางในสนามแม่เหล็ก $\vb{B}$ ที่มีทิศตั้งฉากกับสายไฟ แล้วทำการดึงสายไฟออกด้วยอัตราเร็ว $v$ ในทิศตั้งฉากกับทั้งสายไฟและ $\vb{B}$ ก็จะได้
\[ 
\emf = \int \vb{f}_\txt{mag}\cdot\odif{\vbs{\ell}} = vBh 
\]
แต่เพราะในขณะที่สายไฟมีความเร็ว $v$ กระแสที่เกิดขึ้นก็จะทำให้มีแรงแม่เหล็กต้านไว้ แรงที่ดึงจึงต้องต้านแรงแม่เหล็กนี้ด้วย โดยถ้าสมมติว่าอิเล็กตรอนไหลด้วยอัตราเร็ว $u$ เทียบกับสายไฟ จะได้แรงที่ต้องดึง $\vb{f}_\txt{pull} = uB$ จึงได้ว่างานที่สายไฟนี้ทำต่อประจุเท่ากับ
\[ 
\int \vb{f}_\txt{pull}\cdot\odif{\vbs{\ell}} = uB\ab(\frac{h}{\cos\theta})\sin\theta = vBh 
\]
ดังนั้นจริง ๆ แล้วงานที่เกิดขึ้นนั้นมาจากแรงดึงทั้งหมด ไม่ได้มาจากแรงแม่เหล็ก (ซึ่งก็ไม่น่าแปลกใจเพราะแรงแม่เหล็กไม่ทำงาน) ต่อมาเราจะมาพิสูจน์กฎที่สำคัญในการหา emf จากการสนามแม่เหล็กดังกระบวนการก่อนหน้านี้

พิจารณาสายไฟวงปิดที่เกิดการเคลื่อนที่หรือบิด ทำให้เกิดการเปลี่ยนแปลงฟลักซ์แม่เหล็กที่ผ่านผิวที่กำหนดเส้นขอบโดยสายไฟ เมื่อเวลาผ่านไป $\odif{t}$ ก็จะเกิด ``ริบบิ้น'' ของพื้นที่ส่วนที่เปลี่ยนแปลงขึ้น ก็จะได้
\begin{equation} 
    \odif{\Phi_B} = \int_{\txt{ribbon}}\vb{B}\cdot\odif{\vb{a}}\tag{$\circ$1}\label{circ1} 
\end{equation}
ถ้าพิจารณา $\odif{\vb{a}}$ ที่จุด ๆ หนึ่งโดยให้ความเร็วของอิเล็กตรอน $\vb{w}$ มาจากสองส่วนคือส่วน $\vb{v}$ ที่เป็นความเร็วของสายไฟ และ $\vb{u}$ ที่เป็นความเร็วของกระแส ก็จะได้ว่า
\begin{equation} 
    \odif{\vb{a}} = \vb{v}\odif{t}\times\odif{\vbs{\ell}} = (\vb{w}-\vb{u})\odif{t}\times\odif{\vbs{\ell}} = \vb{w}\odif{t}\times\odif{\vbs{\ell}}\tag{$\circ$2}\label{circ2} 
\end{equation}
นำ (\ref{circ2}) ไปแทนใน (\ref{circ1}) จะได้
\[ 
\odv{\Phi_B}{t} = \int\vb{B}\cdot\ab(\vb{w}\odif{t}\times\odif{\vbs{\ell}}) = -\int\ab(\vb{w}\times\vb{B})\cdot\odif{\vbs{\ell}} = -\emf 
\]
ดังนั้น
\begin{ieqbox}{กฎฟลักซ์แม่เหล็กสำหรับสายไฟเคลื่อนที่} 
    \emf = -\odv{\Phi_B}{t}\label{motionalemf} 
\end{ieqbox}
หมายเหตุ: \emph{กฎนี้เห็นชัดจากการพิสูจน์ว่าต้องเกิดจากการเคลื่อนที่ของสายไฟ ดังนั้นการสับสวิทช์ที่ทำให้วงสายไฟใหญ่ขึ้นจึงไม่ทำให้เกิด emf เป็นอนันต์}

\section{การเหนี่ยวนำแม่เหล็กไฟฟ้า}

\subsection{กฎของ Faraday}

ต่อมา Michael Faraday ได้ทำการทดลองเพิ่มจาก (\ref{motionalemf}) โดยแทนที่จะขยับสายไฟ เขาทำการขยับแม่เหล็กและปรับขนาดของฟลักซ์แม่เหล็กแทน ปรากฏว่า emf ที่เกิดขึ้นก็ยังคงเป็นไปตาม (\ref{motionalemf}) อยู่ดี โดยไม่ต้องขยับสายไฟเลย โดยแรงที่เกิดขึ้นนี้เป็นแรงไฟฟ้าไม่ใช่แรงแม่เหล็ก ดังนั้น
\begin{ieqbox}{กฎของ Faraday (Integral Form)} 
    \emf = \oint \vb{E}\cdot\odif{\vbs{\ell}} = -\int \pdv{\vb{B}}{t}\cdot\odif{\vb{a}} = -\odv{\Phi}{t}\label{faradayint} 
\end{ieqbox}
โดยถ้าใช้ Stokes' theorem ต่อก็จะได้
\[ 
\int\ab(\gd\times\vb{E})\cdot\odif{\vb{a}} = -\int \pdv{\vb{B}}{t}\cdot\odif{\vb{a}} 
\]
หรือก็คือ
\begin{ieqbox}{กฎของ Faraday (Differential Form)} 
    \gd\times\vb{E} = -\pdv{\vb{B}}{t} 
\end{ieqbox}

เราสามารถรวมกฎของ Faraday (\ref{faradayint}) และกฎฟลักซ์แม่เหล็กสำหรับสายไฟเคลื่อนที่ (\ref{motionalemf}) ได้เป็นกฎฟลักซ์แม่เหล็กรวม (หรือบางคนเรียกว่ากฎของ Faraday) ดังนี้:
\begin{eqbox}{กฎฟลักซ์แม่เหล็กรวม} 
    \emf = -\odv{\Phi_B}{t} 
\end{eqbox}
โดยทิศของกระแสอาจจะมึนจึงมีกฎของ Lenz มาช่วยให้คิดทิศของสนามไฟฟ้าเหนี่ยวนำง่ายขึ้น:
\begin{corbox}{กฎของ Lenz} 
    ธรรมชาติต่อต้านการเปลี่ยนแปลงฟลักซ์แม่เหล็กโดยสร้างกระแสไฟฟ้าเหนี่ยวนำในทิศที่จะเกิดสนามแม่เหล็กต้านการเปลี่ยนแปลงของฟลักซ์ 
\end{corbox}

\subsection{สนามไฟฟ้าเหนี่ยวนำ}

เราสามารถสังเกตว่าในกฎของ Faraday ถ้าพิจารณาในปริเวณที่ไม่มีประจุแล้ว
\[ 
\gd\cdot\vb{E} = 0\qq{และ}\gd\times\vb{E} = -\pdv{\vb{B}}{t} 
\]
ซึ่งเหมือนกับสมการของแม่เหล็กสถิต ดังนั้นเราสามารถใช้ทริคต่าง ๆ คล้ายในบทแม่เหล็กสถิต เช่นจะได้ ``กฎ Biot-Savart'' ว่า:
\begin{eqbox}{กฎ Biot-Savart ของสนามไฟฟ้า} 
    \vb{E} = -\frac{1}{4\pi} \pdv{}{t}\int\frac{\vb{B}\times\vus{\rad}}{\rad^2} 
\end{eqbox}
หรือเราอาจจะใช้ (\ref{faradayint}) เพื่อสร้างลูป Amperian ในการคำนวณสนามไฟฟ้าเหนี่ยวนำได้เช่นกัน

\begin{corbox}{ตัวอย่าง}  
    ประจุที่มีความหนาแน่นเชิงเส้น $\lambda$ ถูกนำกาวติดไว้ที่ริมของล้อรัศมี $b$ ที่วางในระนาบ $xy$ และหมุนได้อย่างอิสระ ข้างในล้อมีสนามแม่เหล็กสม่ำเสมอ $\vb{B}_0$ ชี้ในทิศ $+z$ ที่กระจายอยู่ทั่วในทรงกระบอกที่มีศูนย์กลางเดียวกับล้อและมีรัศมี $a < b$ ถ้าเกิดว่าปิดสนามแม่เหล็กนี้แล้วจะเกิดอะไรขึ้นกับล้อ 
\end{corbox}
\begin{soln}
    การปิดสนามนี้จะเกิดการเปลี่ยนแปลงฟลักซ์แม่เหล็กในล้อ จึงจะเกิดสนามไฟฟ้าวนในทิศทวนเข็มนาฬิกา (โดยกฎของ Lenz) เพื่อต้านการเปลี่ยนแปลงของสนามแม่เหล็ก โดยจะได้ทอร์กจากสนามไฟฟ้า ณ เวลาใด ๆ เท่ากับ
    \[ 
    \tau = b\int \odif{F} = b\int E\odif{q} = bE\lambda(2\pi b) 
    \]
    จาก (\ref{ampereint}) จะได้ว่า
    \begin{align*} 
        \int \vb{E}\cdot\odif{\vbs{\ell}} &= -\odv{\Phi_B}{t}\\ 
        E(2\pi b) &= -\pi a^2 \odv{B}{t} 
    \end{align*}
    นำไปแทนในทอร์กจะได้
    \[ 
    \tau = -\lambda\pi a^2b \odv{B}{t} 
    \]
    ดังนั้น
    \[ 
    \Delta L = \int\tau\odif{t} = -\lambda\pi a^2b \int \odv{B}{t}\odif{t} = \lambda\pi a^2b(B_\txt{before} - B_\txt{after}) = \lambda\pi a^2b B_0 
    \]
    จึงได้ว่าไม่ว่าจะปิดสนามแม่เหล็กนี้เร็วแค่ไหน จะเกิดการเปลี่ยนแปลงโมเมนตัมเชิงมุมเท่ากันเสมอ
\end{soln}

ปัญหาหนึ่งของการใช้กฎของ Faraday คือสนามไฟฟ้าเหนี่ยวนำที่เกิดขึ้นนี้จะต้องเกิดจากสนามแม่เหล็กที่เปลี่ยน-แปลง ดังนั้นตามทฤษฎีแล้วเราจึงไม่สามารถคำนวณหาสนามแม่เหล็กที่นำมาใส่ในสมการได้ด้วยสมการเดียวกับบทแม่เหล็กสถิต แต่ในความเป็นจริง ถ้าสนามแม่เหล็กที่เปลี่ยนแปลงนี้เปลี่ยนช้าพอ (เราจะเรียกว่าเปลี่ยนแบบ \emph{quasistatic}) error ที่เกิดขึ้นจากการใช้กฎจากแม่เหล็กสถิตในการคำนวณนี้จะถือว่าน้อยมาก ๆ ในบริเวณที่อยู่ใกล้ ๆ กับแหล่งกำเนิดของการเปลี่ยนแปลงนี้

\subsection{ความเหนี่ยวนำ}

พิจารณาขดลวดสายไฟสองขดลวดที่วางไว้อยู่นิ่ง ถ้าลวดเส้นที่หนึ่งมีกระแส $I_1$ ไหลผ่านจะทำให้เกิดสนามแม่เหล็ก $\vb{B}_1$ และจะเกิดฟลักซ์แม่เหล็ก $\Phi_2$ ผ่านขดลวดที่สอง เนื่องจากกฎ Biot-Savart จะได้ว่า $B_1\propto I_1$ ดังนั้น $\Phi_2\propto I_1$ โดยเราจะเรียกค่าคงที่การแปรผันนี้ว่า\emph{ความเหนี่ยวนำร่วมกัน} (\emph{mutual inductance}: $M_{21}$) ของขดลวดทั้งสอง:
\begin{defbox}{ความเหนี่ยวนำร่วมกัน}
    \begin{equation}
        M_{21} \equiv \frac{\Phi_2}{I_1}\qq{หรือ}\Phi_2 = M_{21}I_1
    \end{equation}
\end{defbox}
พิจารณาการหา $M_{21}$ โดยเราจะเริ่มจาก $\Phi_2$ และใช้ Stokes' theorem:
\[
\Phi_2 = \int \vb{B}_1\cdot\odif{\vb{a}_2} = \int\ab(\gd\times\vb{A}_1)\cdot\odif{\vb{a}_2} = \oint\vb{A}_1\cdot\odif{\vbs{\ell}_2} = \frac{\mu_0I_1}{4\pi}\oint\oint\frac{1}{\rad}\odif{\vbs{\ell}_1}\odif{\vbs{\ell}_2}
\]
ก็จะได้ $M_{21} = M_{12} \equiv M$ และจะได้
\begin{ieqbox}{สูตรของ Neumann}
    M = \frac{\mu_0}{4\pi} \oint\oint\frac{\odif{\vbs{\ell}_1}\cdot\odif{\vbs{\ell}_2}}{\rad}
\end{ieqbox}

จริง ๆ แล้วการเหนี่ยวนำให้เกิดการเปลี่ยนแปลงฟลักซ์แม่เหล็กของสายไฟ ไม่จำเป็นต้องใช้สายไฟสองเส้นก็ได้ เราจึงนิยาม\emph{ความเหนี่ยวนำตัวเอง} (\emph{self inductance}: $L$) หรืออาจจะเรียกสั้น ๆ ว่า\emph{ความเหนี่ยวนำ} ดังนี้
\begin{defbox}{ความเหนี่ยวนำ}
    \begin{equation}
        L \equiv \frac{\Phi_B}{I}\qq{หรือ}\Phi_B = LI
    \end{equation}
\end{defbox}
ความเหนี่ยวนำมีหน่วยเป็น $\unit{H}$ (henry) และโดยกฎของ Faraday ก็จะได้ว่าเมื่อพยายามจะทำให้เกิดการเปลี่ยนแปลงของกระแสในวงจรแบบ quasistatic แล้วจะเกิด emf ในทิศย้อนศร (back emf) เป็นแรงเคลื่อนไฟฟ้าต้านการเปลี่ยนแปลงของกระแส:
\begin{eqbox}{Back Emf}
    \emf_\txt{back} = -L\odv{I}{t}\label{backemf}
\end{eqbox}

ในวงจรหนึ่ง เราสามารถสร้างขดลวดโซลีนอยด์เพื่อให้มีความเหนี่ยวนำตามที่ต้องการได้ เราจะเรียกขดลวดนี้ว่า\emph{ตัวเหนี่ยวนำ} (\emph{inductor})

\subsection{พลังงานในสนามแม่เหล็ก}

จาก (\ref{backemf}) จะเห็นได้ว่าเราจะต้องใช้พลังงานมากกว่าปกติเพื่อที่จะทำให้เกิดกระแสที่ต้องการในตัวเหนี่ยวนำ (หรือในวงจร) โดยงานที่จะต้องต้าน back emf นี้เพื่อให้เกิดกระแส $I$ เท่ากับ
\[
\odv{W}{t} = -\emf I = LI\odv{I}{t}
\]
ดังนั้นจะได้ว่างานที่ต้องใช้ในการสร้างกระแสในตัวเหนี่ยวนำเท่ากับ
\begin{ieqbox}{พลังงานสะสมในตัวเหนี่ยวนำ}
    U = \frac{1}{2}\Phi_B I = \frac{1}{2}LI^2 = \frac{1}{2} \frac{\Phi_B^2}{L}\label{energyinductor}
\end{ieqbox}
เราสามารถเขียน (\ref{energyinductor}) ได้ในอีกรูป พิจารณา
\[
U = \frac{1}{2} \Phi_B I = \frac{1}{2} I \int \vb{B}\cdot\odif{\vb{a}} = \frac{1}{2} I \int \ab(\gd\times\vb{A})\cdot\odif{\vb{a}} = \frac{1}{2} I \oint \vb{A}\cdot\odif{\vbs{\ell}} = \frac{1}{2} \oint\ab(\vb{A}\cdot\vb{I})\odif{\ell}
\]
ขยายมาในสามมิติจะได้
\begin{equation}
    U = \frac{1}{2} \int\ab(\vb{A}\cdot\vb{J})\odif{\tau}
\end{equation}
ใช้ (\ref{amperedif}) ก็จะได้
\begin{align*}
    U &= \frac{1}{2\mu_0} \int_\vol\ab(\vb{A}\cdot\ab(\gd\times\vb{B}))\odif{\tau}\\
    &= \frac{1}{2\mu_0} \ab(\int_\vol \vb{B}\cdot\ab(\gd\times\vb{A})\odif{\tau} - \int_\vol \gd\cdot\ab(\vb{A}\times\vb{B})\odif{\tau}) \\
    &= \frac{1}{2\mu_0}\ab(\int_\vol B^2\odif{\tau} - \oint_{\del\vol} \ab(\vb{A}\times\vb{B})\cdot\odif{\vb{a}})
\end{align*}
เนื่องจากที่ระยะไกล ๆ $\vb{B}$ และ $\vb{A}$ เข้าใกล้ $\vb{0}$ ดังนั้นพจน์หลังจึงหายไป ก็จะได้
\begin{ieqbox}{พลังงานในสนามแม่เหล็ก}
    U = \frac{1}{2\mu_0}\int B^2\odif{\tau}\label{magnetdensity}
\end{ieqbox}
โดยเราสามารถใช้ (\ref{magnetdensity}) และ (\ref{energyinductor}) เพื่อนิยามความเหนี่ยวนำที่เป็นระบบสายไฟเชิงพื้นที่หรือเชิงปริมาตรได้ (การหาฟลักซ์จากระบบเหล่านี้อาจไม่มีนิยามี่ตายตัว) ดังนี้
\begin{defbox}{ความเหนี่ยวนำจากพลังงาน}
    \begin{equation}
        L \equiv \frac{1}{\mu_0I^2}\int B^2 \odif{\tau}
    \end{equation}
\end{defbox}

\section{สมการ Maxwell}

\subsection{ข้อบกพร่องของกฎของ Ampère}

ตอนนี้เรามีสมการสี่สมการที่อธิบายแม่เหล็กไฟฟ้า ดังนี้
\begin{align*}
    \gd\cdot\vb{E} &= \frac{\rho}{\epsilon_0} \tag{กฎของ Gauss}\\
    \gd\cdot\vb{B} &= 0 \\
    \gd\times\vb{E} &= -\pdv{\vb{B}}{t} \tag{กฎของ Faraday}\\
    \gd\times\vb{B} &= \mu_0\vb{J} \tag{กฎของ Ampère} 
\end{align*}
แต่ยังมีข้อบกพร่องอยู่ในกฎของ Ampère เพราะว่ากฎนี้เราอธิบายมาจากกฎ Biot-Savart ซึ่งใช้ได้เฉพาะระบบที่เป็นแม่เหล็กสถิตหรือมีกระแสคงที่ ข้อบกพร้องนี้เห็นได้ชัดถ้าเราพิจารณา divergence ของสมการกฎของ Ampère:
\begin{align*}
    \gd\cdot\ab(\gd\times\vb{B}) &= \mu_0\ab(\gd\cdot\vb{J})\\
    0 &= \mu_0\ab(\gd\cdot\vb{J})
\end{align*}
ซึ่งไม่ได้เป็นจริงเสมอไป

ต่อมา James Clerk Maxwell จึงได้ทำการแก้ข้อบกพร่องนี้โดยอาศัยสมการความต่อเนื่อง:
\[
\gd\cdot\vb{J} + \pdv{\rho}{t} = 0
\]
นำไปเพิ่มในกฎของ Ampère จากสมการด้านบนจะได้
\begin{align*}
    \gd\cdot\ab(\gd\times\vb{B}) &= \mu_0\ab(\gd\cdot\vb{J}) + \mu_0\pdv{\rho}{t}\\
    \gd\cdot\ab(\gd\times\vb{B}) &= \mu_0\ab(\gd\cdot\vb{J}) + \mu_0\eps_0\ab(\gd\cdot\pdv{\vb{E}}{t})
\end{align*}
ดังนั้นถ้าแก้ไขสมการกฎของ Ampere เป็นดังต่อไปนี้ จะได้กฎที่ไร้ข้อขัดแย้ง:
\begin{ieqbox}{กฎของ Ampère-Maxwell}
    \gd\times\vb{B} = \mu_0\vb{J} + \mu_0\eps_0\pdv{\vb{E}}{t}
\end{ieqbox}
และเราจะเรียกพจน์ $\eps_0 (\del\vb{E} / \del t)$ ว่า\emph{กระแสแทนที่} (\emph{displacement current})

โดยนำสมการทั้งสี่มารวมกันทั้งหมดจะได้\emph{สมการ Maxwell} (\emph{Maxwell's equations}):
\begin{lawbox}{สมการ Maxwell}
    การเปลี่ยนแปลงของสนามไฟฟ้า $\vb{E}$ และสนามแม่เหล็ก $\vb{B}$ ทั้งหมดถูกอธิบายได้ด้วยสี่สมการ:
    \begin{align*}
        \gd\cdot\vb{E} &= \frac{\rho}{\epsilon_0} \tag{กฎของ Gauss}\\
        \gd\cdot\vb{B} &= 0 \\
        \gd\times\vb{E} &= -\pdv{\vb{B}}{t} \tag{กฎของ Faraday}\\
        \gd\times\vb{B} &= \mu_0\vb{J} + \mu_0\eps_0\pdv{\vb{E}}{t}\tag{กฎของ Ampère-Maxwell} 
    \end{align*}
    โดยแรงที่ทำให้เกิดกระแสและการเคลื่อนที่ทั้งหมดอธิบายโดยกฎแรง Lorentz (\ref{lorentz})
\end{lawbox}

\section{สมการ Maxwell ในสสาร (TO-DO)}



