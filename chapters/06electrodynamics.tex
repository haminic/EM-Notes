\chapter{พลศาสตร์ไฟฟ้า}
\section{แรงเคลื่อนไฟฟ้า}
\subsection{กฎของ Ohm}

ในการเคลื่อนย้ายประจุให้เกิดกระแสก็จะต้องออกแรง เราจึงมาหาความสัมพันธ์ระหว่างแรงกับกระแสกันก่อน

พิจารณาสายไฟที่มีอิเล็กตรอนอิสระอยู่ $n$ อนุภาคต่อหน่วยปริมาตรและแต่ละอิเล็กตรอนมีมวล $m$ ประจุ $q$ และสมมติมีสนามแรง $\vb{f}$ (ต่อหน่วยประจุ) กระทำอยู่กับทั้งสาย แรง $\vb{f}$ จะทำให้อิเล็กตรอนเคลื่อนที่ด้วยอัตราเร่ง $a$ ก่อนที่จะชนกับอิเล็กตรอนอีกอนุภาคจนทำให้อัตราเร็ว (โดยเฉลี่ยทั้งหมดแล้ว) กลับมาเป็น $0$ อีกครั้ง โดยถ้าสมมติว่าอัตราเร็วของอิเล็กตรอนเนื่องจากความร้อนเท่ากับ $v_\txt{thermal}$ และมีระยะทางเฉลี่ย $\lambda$ ระหว่างการชน เนื่องจาก $v_\txt{thermal}$ มีค่าสูงมาก จึงประมาณได้ว่าความเร่งที่เกิดขึ้นนั้นมีผลน้อยมาก จึงได้เวลาโดยเฉลี่ยก่อนที่จะชนกับอิเล็กตรอนอีกอนุภาคคือ
\[
t = \frac{\lambda}{v_\txt{thermal}}
\]
ก็จะได้ความเร็วเฉลี่ยหรือ\emph{อัตราเร็วลอยเลื่อน} (\emph{drift velocity}) เท่ากับ
\[
v_d = \frac{1}{2}at = \frac{a\lambda}{2v_\txt{thermal}}
\]
ดังนั้นกระแสจึงเท่ากับ
\begin{equation}
    \vb{J} = nq\vb{v}_d = nq\frac{\lambda\vb{a}}{2v_\txt{thermal}} = \ab(\frac{nq\lambda}{2v_\txt{thermal}m})\vb{F} = \ab(\frac{n^2q^2\lambda}{2v_\txt{thermal}m})\vb{f}\label{ohmapprox}
\end{equation}
จะเห็นว่าโดยปกติแล้วสำหรับวัสดุทั่วไป $\vb{J}$ จึงแปรผันตรงกับ $\vb{f}$:
\begin{eqbox}{สมการการแปรผันตรงของกระแสกับแรง}
    \vb{J} = \sigma\vb{f}
\end{eqbox}
โดยที่ $\sigma$ เป็นค่าคงที่ที่เรียกว่า\emph{สภาพนำไฟฟ้า} (\emph{conductivity}) ของสสารนั้น (ถ้าสสารเป็นตัวนำในอุดมคติก็จะมี $\sigma = \infty$) และ $\rho\equiv 1/\sigma$ เรียกว่า\emph{สภาพต้านทาน} (\emph{resistivity}) โดยถ้าแรงที่ใช้เป็นแรงทางไฟฟ้า\underline{เท่านั้น}โดยมีส่วนของแรงแม่เหล็กน้อยมาก ๆ ก็จะได้
\begin{ieqbox}{กฎของ Ohm}
    \vb{J} = \sigma\vb{E}\label{ohme}
\end{ieqbox}
หมายเหตุ: \emph{สมการ (\ref{ohmapprox}) เป็นเพียงการประมาณหยาบ ๆ เท่านั้น จึงไม่สามารถนำมาใช้หา $\sigma$ ได้จริง ๆ และยิ่งไปกว่านั้น ในความเป็นจริงแล้วยังมีวัสดุบางชนิดที่ไม่เป็นไปตามกฎการแปรผันตรงนี้อีกด้วย เราจะเรียกวัสดุที่เป็นไปตามกฎของ Ohm ว่าเป็นวัสดุ Ohmic}

สังเกตว่าในการทำให้ความต่างศักย์มากขึ้น $k$ เท่าระหว่างขั้ว\emph{อิเล็กโทรด} เราจะต้องเพิ่ม $Q$ ไป $k$ เท่า ทำให้ $\vb{E}$ เพิ่ม $k$ เท่าและจาก (\ref{ohme}) จะได้ว่า $\vb{J}$ และ $I$ ก็เพิ่ม $k$ เท่าเช่นกัน ก็จะได้กฎของ Ohm ในอีกรูปแบบ:
\begin{ieqbox}{กฎของ Ohm ในรูปกระแสและความต่างศักย์}
    V = IR
\end{ieqbox}
เมื่อ $R$ เป็นค่าคงที่\emph{ความต้านทาน}ระหว่างสองจุดนั้น (ในการคำนวณหาความต้านทานใช้ (\ref{ohme}) ตามในแต่ละระบบได้เลย)

ในกรณีที่กระแสไหลแบบคงที่ในสสารเนื้อเดียวกันที่เป็นไปตามกฎของ Ohm จาก (\ref{gdjzero}) จะได้ว่า
\begin{equation}
    \gd\cdot\vb{E} = \frac{1}{\sigma}\gd\cdot\vb{J} = 0
\end{equation}
ดังนั้นในบริเวณที่สสารเป็นไปตามกฎของ Ohm ก็จะไม่มีประจุตกค้างอยู่ภายในเลย จึงทำให้สามารถใช้ทริคในการแก้ศักย์และสนามจากสมการ Laplace ได้ตามปกติ

สุดท้าย จาก (\ref{ohmapprox}) เนื่องจากแรงที่ออกนั้นไม่ส่งผลในอัตราเร็วลอยเลื่อนเพิ่มขึ้นเลย ดังนั้นพลังงานส่วนมากจากการชนจะถูกเปลี่ยนเป็นความร้อน โดยถ้ามีประจุไหลต่อเวลาเท่ากับ $I$ โดยศักย์ของประจุลดลง $V$ ก็จะได้
\begin{ieqbox}{กฎการให้ความร้อนของ Joule}
    P = IV = I^2R = \frac{V^2}{R}
\end{ieqbox}

\subsection{แรงเคลื่อนไฟฟ้า}

โดยปกติแล้วในวงจรไฟฟ้าจะมีแรงสองแรงในการทำให้ประจุเคลื่อนที่คือแรงจากแหล่งกำเนิด ($\vb{f}_s$) ซึ่งโดยปกติแล้วแรงนี้จะอยู่แค่ในบริเวณแหล่งกำเนิดเท่านั้น และอีกแรงคือแรงจากสนามไฟฟ้าที่จะเป็นตัวที่ช่วยทำให้กระแสไหลด้วย $I$ คงที่ตลอดทั้งสาย ดังนั้นแรงต่อประจุโดยรวมจะเท่ากับ
\[
\vb{f} = \vb{f}_s + \vb{E}
\]
แต่แรง $\vb{E}$ ที่ช่วยให้กระแสไหลคงที่มาจากไหนล่ะ? เราลองพิจารณาทีละขั้นตอน ดังนี้:
\begin{enumerate}
    \item เมื่อเริ่มต่อสายไฟกับแบตเตอรี่ จะเกิดแรง $\vb{f}_s$ ทำให้เกิดกระแสไหลออก โดยถ้ากระแสในสายไฟเปล่านี้เริ่มไหลไม่คงที่ จะทำให้มีประจุสะสมเกิดขึ้นจึงมี $\vb{E}$ ต้านกระแสส่วนที่เร็วเกินไปและเสริมในส่วนที่ช้าเกินไป
    \item ที่บริเวณตัวต้านทานก็เช่นเดียวกัน จะต้องมีกระแสเท่ากับนอกตัวต้านทาน แต่คราวนี้ประจุจะสะสมไปเรื่อย ๆ จนกว่าสนามไฟฟ้าที่เกิดขึ้นจะมากพอที่จะพลักประจุผ่านตัวต้านทานไปได้ด้วยกระแสเท่ากับข้างนอก (ตาม (\ref{ohme})) โดยกระเกิดประจุสะสมที่ฝั่งหนึ่งของตัวต้านทานก็จะทำให้เกิดประจุสะสมที่ขั้วของแบตเตอรี่ด้วย
    \item อีกขั้วของแบตเตอรี่ก็จะเกิดกระบวนการเช่นเดียวกับ 1. และ 2. แต่ในทิศและขั้วตรงข้าม
\end{enumerate}
เราจึงนิยามผลของแรงทั้งหมดภายในวงจรว่า\emph{แรงเคลื่อนไฟฟ้า}หรือ \emph{emf} (\emph{electromotive force}: $\emf$):
\begin{defbox}{แรงเคลื่อนไฟฟ้า}
    \begin{equation}
        \emf \equiv \oint \vb{f}\cdot\odif{\vbs{\ell}} = \oint \vb{f}_s\cdot\odif{\vbs{\ell}}
    \end{equation}
\end{defbox}
เนื่องจาก $\oint \vb{E}\cdot\odif{\vbs{\ell}} = 0$

หมายเหตุ: \emph{emf นี้นิยามเป็นค่า ณ ขณะหนึ่งเท่านั้น ดังนั้นเมื่อสายไฟขยับ เราาจะใช้ $\odif{\vbs{\ell}}$ เป็นทิศเดียวกับสายไฟจริง ๆ ไม่ต้องคำนึงสถึงความเร็ว}

พิจารณาในสภาวะสมดุลหลังจากต่อแบตเตอรี่: สมมติแหล่งกำเนิดเป็นแบตเตอรี่ไร้ความต้านทาน ($\sigma = \infty$) ก็จะได้ว่าแรงที่ออกในการเคลื่อนประจุเป็น $0$ ดังนั้น $0 = \vb{f} = \vb{f}_s + \vb{E}$ ก็จะได้
\begin{equation}
    V = -\int_{\vb{a}}^{\vb{b}} \vb{E}\cdot\odif{\vbs{\ell}} = \int_{\vb{a}}^{\vb{b}} \vb{f}_s\cdot\odif{\vbs{\ell}} = \oint \vb{f}_s\cdot\odif{\vbs{\ell}} = \emf
\end{equation}
แต่ถ้าแบตเตอรี่นี้มีความต้านทาน $r$ (หมายความว่าถ้าตัดแรง $\vb{f}_s$ ออกแล้วความต่างศักย์ $V_\txt{off} = \int\vb{E}_\txt{off}\cdot\odif{\vbs{\ell}} = Ir$) สมการด้านบนจะไม่เป็นจริง โดยจะได้
\begin{equation}
    V = -\int_{\vb{a}}^{\vb{b}} \vb{E}\cdot\odif{\vbs{\ell}} = \int_{\vb{a}}^{\vb{b}} \ab(\vb{f}_s - \frac{\vb{J}}{\sigma})\cdot\odif{\vbs{\ell}} = \emf + \int_{\vb{a}}^{\vb{b}} \vb{E}_\txt{off}\cdot\odif{\vbs{\ell}} = \emf - V_\txt{off} = \emf - Ir
\end{equation}

\subsection{กฎฟลักซ์แม่เหล็ก}

เราสามารถเหนี่ยวนำเส้นลวดให้เกิด emf ได้โดยอาศัยสนามแม่เหล็ก ซึ่งเป็นวิธีที่\emph{เครื่องกำเนิดไฟฟ้า} (\emph{generator}) ใช้ในการสร้างกระแสไฟฟ้า โดยยกตัวอย่างเช่น ถ้าเราเอาสายไฟรูปสี่เหลี่ยมมุมฉากที่กว้าง $h$ ไปวางในสนามแม่เหล็ก $\vb{B}$ ที่มีทิศตั้งฉากกับสายไฟ แล้วทำการดึงสายไฟออกด้วยอัตราเร็ว $v$ ในทิศตั้งฉากกับทั้งสายไฟและ $\vb{B}$ ก็จะได้
\[
\emf = \int \vb{f}_\txt{mag}\cdot\odif{\vbs{\ell}} = vBh
\]
แต่เพราะในขณะที่สายไฟมีความเร็ว $v$ กระแสที่เกิดขึ้นก็จะทำให้มีแรงแม่เหล็กต้านไว้ แรงที่ดึงจึงต้องต้านแรงแม่เหล็กนี้ด้วย โดยถ้าสมมติว่าอิเล็กตรอนไหลด้วยอัตราเร็ว $u$ เทียบกับสายไฟ จะได้แรงที่ต้องดึง $\vb{f}_\txt{pull} = uB$ จึงได้ว่างานที่สายไฟนี้ทำต่อประจุเท่ากับ
\[
\int \vb{f}_\txt{pull}\cdot\odif{\vbs{\ell}} = uB\ab(\frac{h}{\cos\theta})\sin\theta = vBh
\]
ดังนั้นจริง ๆ แล้วงานที่เกิดขึ้นนั้นมาจากแรงดึงทั้งหมด ไม่ได้มาจากแรงแม่เหล็ก (ซึ่งก็ไม่น่าแปลกใจเพราะแรงแม่เหล็กไม่ทำงาน) ต่อมาเราจะมาพิสูจน์กฎที่สำคัญในการหา emf จากการสนามแม่เหล็กดังกระบวนการก่อนหน้านี้

พิจารณาสายไฟวงปิดที่เกิดการเคลื่อนที่หรือบิด ทำให้เกิดการเปลี่ยนแปลงฟลักซ์แม่เหล็กที่ผ่านผิวที่กำหนดเส้นขอบโดยสายไฟ เมื่อเวลาผ่านไป $\odif{t}$ ก็จะเกิด ``ริบบิ้น'' ของพื้นที่ส่วนที่เปลี่ยนแปลงขึ้น ก็จะได้
\begin{equation}
    \odif{\Phi} = \int_{\txt{ribbon}}\vb{B}\cdot\odif{\vb{a}}\tag{$\circ$1}\label{circ1}
\end{equation}
ถ้าพิจารณา $\odif{\vb{a}}$ ที่จุด ๆ หนึ่งโดยให้ความเร็วของอิเล็กตรอน $\vb{w}$ มาจากสองส่วนคือส่วน $\vb{v}$ ที่เป็นความเร็วของสายไฟ และ $\vb{u}$ ที่เป็นความเร็วของกระแส ก็จะได้ว่า
\begin{equation}
    \odif{\vb{a}} = \vb{v}\odif{t}\times\odif{\vbs{\ell}} = (\vb{w}-\vb{u})\odif{t}\times\odif{\vbs{\ell}} = \vb{w}\odif{t}\times\odif{\vbs{\ell}}\tag{$\circ$2}\label{circ2}
\end{equation}
นำ (\ref{circ2}) ไปแทนใน (\ref{circ1}) จะได้
\[
\odv{\Phi}{t} = \int\vb{B}\cdot\ab(\vb{w}\odif{t}\times\odif{\vbs{\ell}}) = -\int\ab(\vb{w}\times\vb{B})\cdot\odif{\vbs{\ell}} = -\emf
\]
ดังนั้น
\begin{ieqbox}{กฎฟลักซ์แม่เหล็ก}
    \emf = -\odv{\Phi}{t}
\end{ieqbox}
