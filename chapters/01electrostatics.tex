\chapter{ไฟฟ้าสถิต}
\section{สนามไฟฟ้า}
\subsection{แรง Coulomb}
\begin{lawbox}{กฎของ Coulomb}
    สำหรับจุดประจุที่อยู่นิ่ง $q_1$ และ $q_2$ จะได้ว่าแรงที่กระทำต่อประจุ $q_1$ คือ
    \begin{equation}
        \vb{F} = \kem q_1q_2\frac{\vb{r}_1-\vb{r}_2}{\abs{\vb{r}_1-\vb{r}_2}^3} = \kem\frac{q_1q_2}{\rad^2}\vus{\rad}\label{coulomb}
    \end{equation}
    เมื่อ $\vbs{\rad}$ คือเวกเตอร์จาก $q_1$ ไป $q_2$
\end{lawbox}
โดย $\eps_0$ เป็นค่าคงที่ที่เรียกว่า\emph{สภาพยอมในสูญญากาศ} (\emph{permittivity of free space}) เราจะเรียกแรงนี้ว่า\emph{แรง Coulomb}
\subsection{สนามไฟฟ้า}
สังเกตว่าถ้ามีประจุวางไว้อยู่แล้ว เราสามารถนิยาม\emph{สนามไฟฟ้า}ได้ดังนี้:
\begin{defbox}{สนามไฟฟ้า}
\begin{equation}
    \vb{E}(\vb{r})\equiv\frac{\vb{F}(\vb{r})}{q}
\end{equation}
\end{defbox}
และโดยกฎของ Coulomb (\ref{coulomb}) จะได้ว่า
\begin{eqbox}{สนามไฟฟ้าของจุดประจุ}
    \vb{E}(\vb{r})=\kem\sum_{\vb{r}_k\neq\vb{r}}q_k\frac{\vb{r}-\vb{r}_k}{\abs{\vb{r}-\vb{r}_k}^3}=\kem\sum_k\frac{q_k}{\rad_k^2}\vus{\rad}_k
\end{eqbox}
โดยถ้าประจุไม่ดิสครีตแต่กระจายตัวอย่างต่อเนื่องด้วยความหนาแน่นประจุ $\rho(\vb{r})$ จะได้ว่า
\begin{eqbox}{สนามไฟฟ้าของประจุต่อเนื่อง}
    \vb{E}(\vb{r})=\kem\int\frac{\rho(\vb{r}')}{\rad^2}\vus{\rad}\odif{\tau'}
\end{eqbox} 
เมื่อ $\odif{\tau'}=\odif[order=3]{r'}$ และในทำนองเดียวกันกับความหนาแน่นเชิงเส้น $\lambda$ และความหนาแน่นเชิงพื้นที่ $\sigma$

\section{Divergence และ Curl ของสนามไฟฟ้าสถิต}
\subsection{ฟลักซ์ไฟฟ้าและกฎของ Gauss}
\begin{defbox}{ฟลักซ์ไฟฟ้า}
    ฟลักซ์ของ $\vb{E}$ ที่ผ่านผิว $\sur$ คือ
    \begin{equation}
        \Phi_E\equiv\int_\sur\vb{E}\cdot\odif{\vb{a}}
    \end{equation}
\end{defbox}
พิจารณาพื้นผิวปิด $\sur$ ที่มีจุดประจุ $q$ อยู่ภายในและพื้นที่เล็ก ๆ $\odif{\tau}$ บน $\sur$ โดยมี $\vbs{\rad}$ เป็นเวกเตอร์จาก $q$ มายัง $\odif{a}$ และ $\odif{a'}$ เป็นภาพฉายของ $\odif{a}$ มาตั้งฉากกับ $\vbs{\rad}$ จะได้
\[
\vb{E}\cdot \odif{\vb{a}}=\kem\frac{q}{\rad^2}\odif{a}\cos\theta=\kem\frac{q}{\rad^2}\odif{a'}=\frac{q}{4\pi\eps_0}\odif{\Omega}
\]
เมื่อ $\odif{\Omega}$ คือมุมสเตอเรเดียนเทียบกับตำแหน่งของประจุ $q$ ดังนั้นฟลักซ์ไฟฟ้าจาก $q$ ที่ผ่านพื้นผิว $\sur$ เท่ากับ
\begin{equation}
\Phi_E^{(q\,\txt{in})}=\oint_\sur\vb{E}\cdot\odif{\vb{a}}=\frac{q}{4\pi\eps_0}\oint_\sur\odif{\Omega}=\frac{q}{4\pi\eps_0}\cdot4\pi=\frac{q}{\eps_0}
\end{equation}
โดยทำในทำนองเดียวกันจะเห็นว่าถ้า $q$ อยู่นอก $\sur$ แล้ว $\vb{E}\cdot\odif{\vb{a}}=(\cst)\odif{\Omega}$ const จะมีคู่ของมันที่เครื่องหมายตรงข้ามในอีกฝั่งของ $\sur$ จึงทำให้ตัดกันหมด ดังนั้นในกรณีจุดประจุ $q$ อยู่นอก $\sur$ จะได้ว่าฟลักซ์ไฟฟ้า:
\begin{equation}
\Phi_E^{(q\,\txt{out})}=\oint_\sur\vb{E}\cdot\odif{\vb{a}}=0
\end{equation}
ดังนั้นจึงได้
\begin{lawbox}{กฎของ Gauss (Integral form)}
    สำหรับทุกพื้นผิวปิดจะได้ว่าฟลักซ์ไฟฟ้าที่ผ่านผิวนั้นเท่ากับ
    \begin{equation}
        \Phi_E=\oint\vb{E}\cdot\odif{\vb{a}}=\frac{Q_\txt{enc}}{\eps_0}\label{gauss}
    \end{equation}
\end{lawbox}

\subsection{Divergence และ Curl ของสนามไฟฟ้าสถิต}
พิจารณาใช้ divergence theorem ($\oint_{\del\vol}\vb{F}\cdot\odif{\vb{a}}=\int_\vol\gd\cdot\vb{F}\odif{\tau}$) บนกฎของ Gauss (\ref{gauss}) จะได้ว่า
\begin{align*}
\int_\vol\gd\cdot\vb{E}\odif{\tau}=\oint_{\del\vol}\vb{E}\cdot\odif{\vb{a}}=\frac{Q_\txt{enc}}{\eps_0}&=\int_\vol\frac{\rho}{\eps_0}\odif{\tau}\\
\int_\vol\ab(\gd\cdot\vb{E}-\frac{\rho}{\eps_0})\odif{\tau}&=0
\end{align*}
เนื่องจากเป็นจริงทุกปริมาตร $\vol$ ดังนั้น
\[
\gd\cdot\vb{E}-\frac{\rho}{\eps_0}=0
\]
ก็จะได้
\begin{lawbox}{กฎของ Gauss (Differential form)}
    \begin{equation}
        \gd\cdot\vb{E} = \frac{\rho}{\eps_0}\label{gaussde}
    \end{equation}
\end{lawbox}
และเนื่องจาก curl ของจุดประจุเท่ากับ $\vb{0}$ ดังนั้นจึงได้ว่า curl ของสนาม $\vb{E}$ สถิตใด ๆ จึงเท่ากับ $\vb{0}$ ด้วย
\begin{ieqbox}{กฎของ Faraday สำหรับสนามไฟฟ้าสถิต}
    \gd\times\vb{E} = \vb{0}\label{curlezero}
\end{ieqbox}

\section{ศักย์ไฟฟ้า}
\subsection{นิยามศักย์ไฟฟ้า}
เนื่องจาก $\gd\times\vb{E}=0$ โดย Stokes' theorem จะได้ว่า $\int\vb{E}\cdot\odif{\vb{l}}$ ไม่ขึ้นอยู่กับเส้นทาง เราจึงสามารถนิยามฟังก์ชันที่ขึ้นอยู่กับอินทิกรัลของสนามไฟฟ้า ณ ตำแหน่งใด ๆ ได้:
\begin{defbox}{ศักย์ไฟฟ้า}
    ให้ $\mathcal{O}$ เป็นจุดอ้างอิง เราสามารถนิยาม\emph{ศักย์ไฟฟ้า} $V(\vb{r})$ ที่จุด $\vb{r}$ คือ
    \begin{equation}
        V(\vb{r})\equiv-\int_\mathcal{O}^\vb{r}\vb{E}\cdot\odif{\vb{l}}
    \end{equation}
    ซึ่งโดยปกติแล้วเราจะนิยามศักย์ไฟฟ้าให้ $\eval{V}_{r\to\infty}=0$
\end{defbox}
โดยจะได้\emph{ความต่างศักย์}ระหว่าง $\vb{a}$ และ $\vb{b}$ คือ
\begin{equation}
    V(\vb{b})-V(\vb{a})=-\int_\vb{a}^\vb{b}\vb{E}\cdot\odif{\vb{l}}
\end{equation}
และจาก
\[
\int_\vb{a}^\vb{b}(\gd V)\cdot\odif{\vb{l}}=V(\vb{b})-V(\vb{a})=-\int_\vb{a}^\vb{b}\vb{E}\cdot\odif{\vb{l}}
\]
จะได้ว่า
\begin{eqbox}{ศักย์ไฟฟ้าในรูป Gradient}
    \vb{E}=-\gd V\label{gdephi}
\end{eqbox}
อีกสมการหนึ่งที่สำคัญที่ได้จากศักย์ไฟฟ้าโดยนำสมการ (\ref{gdephi}) ไปแทนใน (\ref{gaussde}) จะได้
\begin{lawbox}{สมการ Poisson}
    สำหรับสนามศักย์ไฟฟ้า $ V$:
    \begin{equation}
        \nabla^2 V = -\frac{\rho}{\eps_0}\label{poissoneq}
    \end{equation}
    โดยถ้า $\rho=0$ จะได้\emph{สมการ Laplace}
    \begin{equation}
        \nabla^2 V = 0\label{laplaceeq}
    \end{equation}
\end{lawbox}
โดยสามารถหา $V$ ของจุดประจุ $q$ ได้จากกฎของ Coulomb (\ref{coulomb}):
\[
V(\vb{r})=-\int_\infty^\vb{r}\vb{E}(\vb{r}')\cdot\odif{\vb{l}'}=-\int_\infty^r\kem\frac{q}{(r')^2}\odif{r'}
\]
ก็จะได้ว่า
\begin{eqbox}{ศักย์ไฟฟ้าของจุดประจุ}
     V=\kem\frac{q}{\rad}
\end{eqbox}
ในทำนองเดียวกันกับสนามไฟฟ้า เราสามารถหาศักย์ไฟฟ้าที่ตำแหน่ง $\vb{r}$ ที่เกิดจากประจุที่กระจายแบบต่อเนื่องด้วยความหนาแน่น $\rho$ ได้ดังนี้:
\begin{eqbox}{ศักย์ไฟฟ้าของประจุต่อเนื่อง}
     V(\vb{r})=\kem\int\frac{\rho(\vb{r}')}{\rad}\odif{\tau'}\label{potentialcont}
\end{eqbox}
\subsection{สภาวะขอบเขต}
ต่อมาจะมาดูสมบัติของ $\vb{E}$ และ $V$ ในบริเวณแผ่นประจุบาง ๆ ที่มีความหนาแน่นประจุเชิงพื้นที่ $\sigma$
\begin{enumerate}
    \item พิจารณาผิว Gaussian ทรงกระบอกบางที่บางมากจนฟลักซ์ไฟฟ้าที่ผ่านบริเวณผิวข้างเท่ากับ $0$ ที่คลุมบริเวณเล็ก ๆ ของแผ่นประจุ จะได้ว่า
    \[
    E^\perp_\txt{above}-E^\perp_\txt{below}=\frac{\sigma}{\eps_0}
    \]
    จะเห็นได้ว่าส่วนของ $\vb{E}$ ที่ตั้งฉากกับแผ่นประจุจะเกิดความไม่ต่อเนื่องแบบกระโดดด้วยผลต่าง $\frac{\sigma}{\eps_0}$
    \item พิจารณาอินทิกรัลเส้นรูปสี่เหลี่ยมผืนผ้าเล็ก ๆ ที่ตั้งฉากกับแผ่นประจุ จาก (\ref{curlezero}) และ Stokes' theorem จะได้ว่า $\oint\vb{E}\cdot\odif{\vb{l}}=0$ ดังนั้น
    \[
    E^\|_\txt{above}-E^\|_\txt{below}=0
    \]
    จะเห็นได้ว่าส่วนของ $\vb{E}$ ที่ขนานกับแผ่นประจุจะยังต่อเนื่องเมื่อผ่านแผ่นประจุ
    \item พิจารณาจุด $\vb{a}$ และ $\vb{b}$ ที่อยู่ใกล้กันมาก ๆ แต่ $\vb{b}$ อยู่ด้านบนแผ่นส่วน $\vb{a}$ อยู่ด้านล่างแผ่น จะได้ว่า
    \[
    V_\txt{above}-V_\txt{below}=V(\vb{b})-V(\vb{a})=-\int_\vb{a}^\vb{b}\vb{E}\cdot\odif{\vb{l}}=0
    \]
    ดังนั้น $V$ ต่อเนื่องเมื่อผ่านแผ่นประจุ
\end{enumerate}
จึงสรุปได้ดังนี้:
\begin{lawbox}{สภาวะขอบเขตของ $\vb{E}$ และ $V$ เมื่อผ่านแผ่นประจุบาง}
บนแผ่นประจุที่มีความหนาแน่นเชิงพื้นที่ $\sigma$ จะได้ว่า
\begin{equation}
    V_\txt{above}=V_\txt{below}\label{boundarystart}
\end{equation}
และ
\begin{equation}
    \vb{E}_\txt{above}-\vb{E}_\txt{below}=\frac{\sigma}{\eps_0}\vu{n}
\end{equation}
เมื่อ $\vu{n}$ คือเวกเตอร์หนึ่งหน่วยที่ตั้งฉากกับแผ่นประจุที่ชี้จากด้านล่างไปด้านบน หรือเขียนอีกอย่างหนึ่งได้ว่า
\begin{equation}
    \pdv{V_\txt{above}}{n}-\pdv{V_\txt{below}}{n}=-\frac{\sigma}{\eps_0}\label{boundaryend}
\end{equation}
\end{lawbox}

\section{งานและพลังงาน}
\subsection{พลังงานศักย์ไฟฟ้า}
เนื่องจาก $\gd\times\vb{F}=\gd\times q\vb{E}=0$ ดังนั้นแรง Coulomb จึงเป็นแรงอนุรักษ์ซึ่งมีลักษณะคล้ายกับแรงโน้มถ่วงด้วย จึงหาพลังงานศักย์ของจุดประจุ $2$ ตัวได้คล้ายกัน
\begin{eqbox}{พลังงานศักย์ไฟฟ้าสำหรับจุดประจุ}
    U=\kem\frac{q_1q_2}{\rad}
\end{eqbox}
และจาก (\ref{gdephi}) ยังได้อีกว่า
\begin{eqbox}{ศักย์ไฟฟ้าและพลังงานศักย์}
    V=\frac{U}{q}
\end{eqbox}
ต่อมาจะมาหาพลังงานศักย์ไฟฟ้าของระบบประจุที่อยู่ภายใต้อิทธิพลของสนามภายนอก $\vb{E}_\txt{ext}$ โดยพิจารณาการหาผลต่างของพลังงานศักย์ในการนำจุดประจุจาก $\infty$ มาวางทีจะตัว จะได้ผลต่างพลังงานศักย์ของประจุที่ $k$ เป็นดังนี้:
\[
\Delta U_k=q_k V_\txt{ext}(\vb{r}_k)
\]
เนื่องจากนิยามให้ $\eval{U}_{r\to\infty}=\eval{qV}_{r\to\infty}=0$ ก็จะได้
\begin{equation}
U_\txt{ext}=\sum_k\Delta U_k=\sum_kq_k V_\txt{ext}(\vb{r}_k)
\end{equation}
หรือขยายมาในกรณีต่อเนื่องก็คือ
\begin{ieqbox}{พลังงานศักย์ไฟฟ้าจากสนามภายนอก}
    U_\txt{ext}=\int\rho V_\txt{ext}\odif{\tau}
\end{ieqbox}
ส่วนพลังงานศักย์ไฟฟ้าที่เกิดจากระบบเองหาได้โดยการพิจารณาเอาจุดประจุจาก $\infty$ มาวางเช่นเดียวกัน จะได้ประจุตัวที่ $k$ มีพลังงานศักย์
\[
U_k=\sum_{k'<k}q_k V_{k'}(\vb{r}_k)=\kem\sum_{k'<k}\frac{q_kq_{k'}}{\rad_{kk'}}
\]
ดังนั้นโดยใช้ความสมมาตร
\begin{equation}
U=\kem\sum_{k}\sum_{k'<k}\frac{q_kq_{k'}}{\rad_{kk'}}=\frac{1}{2}\kem\sum_{k\neq k'}\frac{q_kq_{k'}}{\rad_{kk'}}\label{disintenergy}
\end{equation}
แต่ในกรณีต่อเนื่องเราไม่จำเป็นต้องสนใจเงื่อนไข $k\neq k'$ เพราะอินทิกรัลลู่เข้าและส่วนที่มาจาก $k = k'$ สามารถมองเป็นส่วนพลังงานที่มาจากประจุที่ใกล้กันมาก ๆ ได้ จึงได้ว่า
\[
U=\frac{1}{2}\kem\iint\frac{\rho(\vb{r})\,\rho(\vb{r}')}{\rad}\odif{\tau',\tau}
\]
เนื่องจาก $\kem\int\frac{\rho(\vb{r}')\odif{\tau'}}{\rad}= V(\vb{r})$ ดังนั้น
\begin{ieqbox}{พลังงานศักย์ไฟฟ้าจากสนามภายใน}
    U=\frac{1}{2}\int\rho V\odif{\tau}\label{intpotential}
\end{ieqbox}

\subsection{พลังงานจากสนามไฟฟ้า}
ต่อมาพิจารณาพลังงานศักย์ภายในของระบบอีกแบบโดยแทนสมการ Poisson (\ref{poissoneq}) เข้าไปใน (\ref{intpotential}) โดยอินทิเกรต บนปริมาตร $\vol$ ที่ใหญ่มาก ๆ จน $\vb{E}$ ที่ผิวของ $\vol$ เข้าใกล้ศูนย์ จะได้ว่า
\[
U=-\frac{\eps_0}{2}\int_\vol V\nabla^2 V\odif{\tau}=-\frac{\eps_0}{2}\int_\vol V\ab(\gd\cdot(\gd V))\odif{\tau}=\frac{\eps_0}{2}\int_\vol V\ab(\gd\cdot\vb{E})\odif{\tau}
\]
โดย Chain Rule: $\gd\cdot( V\vb{F})=\gd V\cdot\vb{F}+ V(\gd\cdot\vb{F})$ นำไปแทนต่อ จากนั้นใช้ divergence theorem จะได้ว่า
\begin{align*}
    U&=\frac{\eps_0}{2}\int_\vol V\ab(\gd\cdot\vb{E})\odif{\tau}\\
    &=\frac{\eps_0}{2}\ab(\int_\vol\gd\cdot( V\vb{E})\odif{\tau}-\int_\vol(\gd V)\cdot(\vb{E})\odif{\tau})\\
    &=\frac{\eps_0}{2}\ab(\oint_{\del\vol} V\vb{E}\cdot\odif{\vb{a}}+\int_\vol E^2\odif{\tau})
\end{align*}
แต่จาก $\vb{E}$ ที่ขอบเป็น $0$ พจน์แรกจึงหายไป ดังนั้น
\begin{ieqbox}{พลังงานจากสนามไฟฟ้า}
    U=\int u(\vb{r})\odif{\tau}\qq{เมื่อ} u(\vb{r})=\frac{\eps_0}{2}E^2(\vb{r})\label{energydensity}
\end{ieqbox}
โดยเราจะเรียก $u$ ว่า\emph{ความหนาแน่นพลังงานสนามไฟฟ้า} (\emph{energy density of the electric field})

\emph{แต่คำถามคือ}: ทำไมสมการ (\ref{disintenergy}) ทำให้พลังงานศักย์เป็นลบได้แต่ (\ref{energydensity}) จึงเป็นบวกเสมอ? เหตุผลก็คือ (\ref{disintenergy}) ยังไม่ได้รวมพลังงานในการสร้างจุดประจุตั้งแต่แรก (ถ้ารวมด้วยจะทำให้เป็น $\infty$) ดังนั้นถ้าจะหาพลังงานของระบบที่เป็นจุดประจุ ถ้าใช้ (\ref{disintenergy}) จะสมเหตุสมผลกว่า

ต่อมาเรามาพิจารณาพลังงานศักย์ไฟฟ้าเนื่องจากอิทธิพลของทั้งสนามภายนอกและภายใน:
\[
U=U_\txt{int}+U_\txt{ext}=\frac{1}{2}\int\rho V_\txt{int}\odif{\tau}+\int\rho V_\txt{ext}\odif{\tau}
\]
หาพจน์ฝั่งขวาโดยทำคล้าย ๆ (\ref{energydensity}):
\begin{align*}
    \int_\vol\rho V_\txt{ext}\odif{\tau}&=-\eps_0\int_\vol V_\txt{ext}\ab(\gd\cdot(\gd V_\txt{int}))\odif{\tau}\\
    &=-\eps_0\ab(\cancel{\oint_{\del\vol}- V_\txt{ext}\vb{E}_\txt{int}\cdot\odif{\vb{a}}}-\int_\vol\vb{E}_\txt{ext}\cdot\vb{E}_\txt{int}\odif{\tau})\\
    &=\eps_0\int_\vol\vb{E}_\txt{ext}\cdot\vb{E}_\txt{int}\odif{\tau}
\end{align*}
นำไปแทนในสมการ $U$ และใช้ร่วมกับ (\ref{energydensity}) จะได้
\begin{equation}
U=\int u(\vb{r})\odif{\tau}\qq{เมื่อ}u(\vb{r})=\frac{\eps_0}{2}\ab\Big(E_\txt{int}^2(\vb{r})+2\vb{E}_\txt{int}(\vb{r})\cdot\vb{E}_\txt{ext}(\vb{r}))
\end{equation}
ซึ่งจริง ๆ แล้วเหมือนกับ (\ref{energydensity}) เลย โดยบวกเข้าลบออกด้วย $E_\txt{ext}^2(\vb{r})$ ใน $u(\vb{r})$ และให้ $\vb{E}=\vb{E}_\txt{int}+\vb{E}_\txt{ext}$ จะได้
\[
U=\frac{\eps_0}{2}\int E^2(\vb{r})\odif{\tau}-\underbrace{\frac{\eps_0}{2}\int E_\txt{ext}^2(\vb{r})\odif{\tau}}_{\cst}
\]
เนื่องจากพจน์ด้านหลังเป็นค่าคงที่ เราจึงสามารถให้พจน์นั้นเป็นค่าอ้างอิงได้ จึงได้ว่าเราสามารถใช้ (\ref{energydensity}) ได้ในทุกกรณี เพียงแค่ต้องรวม $\vb{E}_\txt{ext}$ ไปด้วย:
\begin{equation}
    U'=\frac{\eps_0}{2}\int E^2(\vb{r})\odif{\tau}
\end{equation}
สุดท้ายจะเป็นพิสูจน์ทฤษฎีบท:
\begin{ieqbox}{Green's Reciprocity Theorem}
    \int\rho_1 V_2\odif{\tau}=\int\rho_2 V_1\odif{\tau}\label{recipstatic}
\end{ieqbox}
ทฤษฎีบทนี้หมายความว่าพลังงานศักย์ไฟฟ้าในระบบ $1$ ที่เกิดจากระบบ $2$ มีค่าเท่ากับพลังงานศักย์ไฟฟ้าในระบบ $2$ ที่เกิดจากระบบ $1$ ซึ่งก็ไม่แปลกเพราะแรง Coulomb เป็นแรงที่เป็นไปตามกฎข้อที่ $3$ ของนิวตัน แต่จะมาพิสูจน์กันดังนี้:
\begin{proof}
พิจารณาปริมาตร $\vol$ ที่ใหญ่มาก ๆ
\begin{align*}
    \int_\vol\vb{E}_1\cdot\vb{E}_2\odif{\tau}&=-\int_\vol\gd V_1\cdot\vb{E}_2\odif{\tau}\\
    &=-\ab(\int_\vol\gd\cdot( V_1\vb{E}_2)\cdot\odif{\vb{a}}-\int_\vol V_1\gd\cdot\vb{E}_2\odif{\tau})\\
    &=-\ab(\cancel{\oint_{\del\vol} V_1\vb{E}_2\cdot\odif{\vb{a}}}-\frac{1}{\eps_0}\int_\vol V_1\rho_2\odif{\tau})\\
    &=\frac{1}{\eps_0}\int_\vol V_1\rho_2\odif{\tau}
\end{align*}
ในทำนองเดียวกัน:
\[
\int_\vol\vb{E}_1\cdot\vb{E}_2\odif{\tau}=\frac{1}{\eps_0}\int_\vol V_2\rho_1\odif{\tau}
\]
ดังนั้น $\int\rho_1 V_2\odif{\tau}=\int\rho_2 V_1\odif{\tau}$ ตามต้องการ
\end{proof}
\section{ตัวนำและความจุไฟฟ้า}
\subsection{ตัวนำไฟฟ้า}
ในวัตถุที่เป็น\emph{ฉนวนไฟฟ้า} (หรือ\emph{ไดอิเล็กทริก}) อิเล็กตรอนจะเคลื่อนที่ภายในบริเวณอะตอมของมัน แต่ใน\emph{ตัวนำไฟฟ้า} จะมีอิเล็กตรอนจำนวนหนึ่งเคลื่อนที่ได้อย่างอิสระในเนื้อตัวนำ (ในตัวนำที่เป็นของเหลวเช่นน้ำเกลือจะเป็นไอออนอย่าง $\mathrm{Na}^+$ และ $\mathrm{Cl}^-$ ที่เคลื่อนที่ได้อย่างอิสระแทน) โดยตัวนำอุดมคติหมายถึงตัวนำที่มีประจุอิสระไม่จำกัด ซึ่งโลหะจะเป็นตัวนำที่ใกล้เคียงกับตัวนำอิสระพอที่จะใช้การประมาณดังต่อไปนี้ได้:
\begin{lawbox}{สมบัติของตัวนำไฟฟ้าอุดมคติ}
    ตัวนำไฟฟ้าในสภาวะสมดุลจะต้องไม่มีประจุเคลื่อนที่ในเนื้อตัวนำ จึงสามารถตั้งข้อสมมติเกี่ยวกับสนามไฟฟ้าภายในเนื้อตัวนำได้ว่า
    \begin{equation}
        \vb{E} =\vb{0}\label{condasmp}
    \end{equation}
    ซึ่งจะเรียกว่า \emph{electric field screening effect} สังเกตว่าจากกฎของ Gauss จะแปลว่าไม่มีประจุอยู่ภายในเนื้อตัวนำ ประจุทั้งหมดจะรวมกันที่ผิวเท่านั้น
    
    โดย (\ref{condasmp}) สามารถเขียนได้ในอีกรูปคือ
    \begin{equation}
         V = \cst
    \end{equation}
    อีกสมบัติหนึ่งคือจาก (\ref{boundarystart}) ถึง (\ref{boundaryend}) และ (\ref{condasmp}) จะได้ว่าสนามไฟฟ้าที่ผิวตัวนำจะตั้งฉากกับผิวเสมอและมีความสัมพันธ์กับความหนาแน่นประจุดังนี้
    \begin{equation}
         \sigma=\eps_0E_\txt{out}=-\eps_0\pdv{V}{n}
    \end{equation}
\end{lawbox}
สถานการณ์หนึ่งที่น่าสนใจคือเมื่อมี ``\emph{โพรง}" อยู่ในเนื้อตัวนำ โพรงนี้จะเปรียบเสมือนว่าไม่โดนผลกระทบจากสนามไฟฟ้าด้านนอกตัวนำเลย ซึ่งสามารถพิสูจน์ได้โดยใช้ทฤษฎีบท uniqueness ในบทถัดไป โดยจะเรียกตัวนำที่กันสนามภายนอกนี้ว่า \emph{Faraday's cage} (ในทางกลับกัน สนามด้านนอกตัวนำจะไม่โดนผลกระทบจากประจุด้านในโพรง) โดยถ้าในโพรงไม่มีประจุ จะได้ว่าสนามไฟฟ้าในโพรงเป็น $0$ (พิสูจน์กรณีนี้ไม่ยาก ได้จากการสังเกตว่าถ้ามีสนามไฟฟ้าจะต้องมีเส้นแรงไฟฟ้าที่ลากจากผิวไปผิวบนโพรง ถ้าสร้างเส้นทางปิดในการอินทิเกรตบนเส้นแรงนั้นจะได้ผลลัพธ์ไม่เป็น $0$ ซึ่งจาก (\ref{curlezero}) เกิดข้อขัดแย้ง) แต่ถ้านำประจุ $Q$ ไว้ในโพรง โดยกฎของ Gauss จะได้ว่าประจุที่อยู่บนผิวของโพรงจะต้องรวมได้ $-Q$

และยิ่งไปกว่านั้น ถ้าโพรงดังกล่าวอยู่ในตัวนำทรงกลมที่ไม่มีประจุ (ประจุรวมเป็น $0$) สนามไฟฟ้าด้านนอกทรงกลมนั้นจะเปรียบเสมือนสนามไฟฟ้าของตัวนำทรงกลมประจุ $Q$ ทั้งนี้เป็นเพราะมัน ``\emph{เป็นไปได้}" ที่ประจุด้านในจะเรียงตัวให้ประจุที่ผิวของโพรงกับประจุ $Q$ ในโพรงหักล้างกันหมดด้านนอกโพรง และเมื่อมีวิธีการเรียงตัวหนึ่งที่เป็นไปได้ที่ทำให้สนามในเนื้อตัวนำเป็น $\vb{0}$ ปรากฏว่า (ซึ่งจะพิสูจน์ในบทถัดไป) วิธีการจัดเรียงประจุนั้นจะเป็นวิธีเดียวเท่านั้น

\subsection{แรงบนตัวนำไฟฟ้า}
ต่อมาพิจารณาแรงที่กระทำต่อผิวตัวนำ $\odif{a}$ ก้อนเล็ก ๆ จะได้ว่าสนามไฟฟ้าในบริเวณนั้นมาจากสองส่วนคือ $\vb{E}_\txt{other}$ มาจากประจุอื่น ๆ นอกบริเวณ $\odif{a}$ และ $\vb{E}_\txt{self}$ มาจาก $\odif{a}$ เอง โดยสนามด้านบนและด้านล่างของ $\vb{E}_\txt{self}$ คือ $\sigma/2\eps_0$ และ $-\sigma/2\eps_0$ ตามลำดับ (เพราะสนามนี้ดูในบริเวณที่ใกล้ $\odif{a}$ มาก ๆ จนเปรียบเสมือน $\odif{a}$ เป็นผิวราบอนันต์) ดังนั้นจะได้
\begin{align*}
    \vb{E}_\txt{above}&=\vb{E}_\txt{other}+\frac{\sigma}{2\eps_0}\vu{n}\\
    \vb{E}_\txt{below}&=\vb{E}_\txt{other}-\frac{\sigma}{2\eps_0}\vu{n}
\end{align*}
ดังนั้น
\[
\vb{E}_\txt{other}=\frac{1}{2}\ab(\vb{E}_\txt{above}+\vb{E}_\txt{below})
\]
ก็จะได้แรงที่กระทำต่อ $\odif{a}$ คือ
\[
\odif{\vb{F}}=\sigma\odif{a}\cdot\vb{E}_\txt{other}
\]
ดังนั้นแรงต่อหน่วยพื้นที่ $\vb{f}=\odif{\vb{F}}/\odif{a}$ คือ
\begin{ieqbox}{แรงต่อพื้นที่บนแผ่นประจุ}
    \vb{f}=\sigma\vb{E}_\txt{average}=\frac{1}{2}\sigma\ab(\vb{E}_\txt{above}+\vb{E}_\txt{below})
\end{ieqbox}
ซึ่งจริง ๆ แล้วใช้ได้กับแผ่นประจุทุกกรณี แต่ในกรณีตัวนำ:
\begin{eqbox}{แรงต่อพื้นที่บนผิวตัวนำ}
    \vb{f}=\frac{\sigma^2}{2\eps_0}\vu{n}
\end{eqbox}
จะได้ว่าเมื่อด้านนอกตัวนำมีสนาม $\vb{E}$ แล้ว\emph{ความดันไฟฟ้าสถิต} (\emph{electrostatic pressure}: $P$) เป็นดังนี้
\begin{eqbox}{ความดันไฟฟ้าสถิตบนผิวตัวนำ}
    P=\frac{\eps_0}{2}E^2
\end{eqbox}

\subsection{ความจุไฟฟ้า}
เมื่อมีตัวนำสองตัวโดยตัวหนึ่งมีประจุ $+Q$ และอีกตัว $-Q$ เนื่องจากเมื่อ $Q$ เพิ่มขึ้นจำนวน $k$ เท่า จะได้ว่าทำให้ $\sigma$ บนทั้งสองประจุเพิ่มขึ้นเป็น $k$ เท่าเช่นกัน (เพราะมีการจัดเรียงแบบเดียวเท่านั้นที่ทำให้เนื้อตัวนำมี $\vb{E}=\vb{0}$ ซึ่งจะพิสูจน์ในบทถัดไป) ส่งผลให้ $\vb{E}$ เพิ่มเป็น $k$ เท่า จึงทำให้ความต่างศักย์ $V=V_{+}-V_-$ ก็เพิ่มขึ้นเป็น $k$ เท่าด้วย จึงสรุปได้ว่า $Q\propto V$ ดังนั้นเราสามารถนิยามค่าคงที่การแปรผันนี้ว่า\emph{ความจุไฟฟ้า} (\emph{capacitance}: $C$) ดังนี้
\begin{defbox}{ความจุไฟฟ้า}
    \begin{equation}
        C\equiv\frac{Q}{V}
    \end{equation}
\end{defbox}
ส่วนความจุไฟฟ้าของตัวนำตัวเดียว (self-capacitance) คือให้จินตนาการว่ามีตัวนำเปลือกทรงกลมที่มีรัศมีใหญ่มาก ๆ หรือก็คือให้ใช้ $V$ เป็น $V$ ของตัวนำโดยมีจุดอ้างอิงเป็น $\infty$

สุดท้าย งานในการชาร์จตัวเก็บประจุหาได้โดยรวมงานในการย้ายประจุ $\odif{q}$ จากฝั่งลบมาฝั่งบวก:
\[
\odif{W}=V\odif{q}=\frac{q}{C}\odif{q}
\]
ดังนั้นงานในการชาร์จประจุจาก $0$ มาเป็น $Q$ (หรือก็คือพลังงานสะสมในตัวเก็บประจุ) เท่ากับ
\begin{ieqbox}{พลังงานสะสมในตัวเก็บประจุ}
    U=\frac{1}{2}QV=\frac{1}{2}\frac{Q^2}{C}=\frac{1}{2}CV^2
\end{ieqbox}

\subsection{การต่อตัวเก็บประจุ}
พิจารณาการต่อตัวเก็บประจุ $C_1$ และ $C_2$ แบบ\emph{อนุกรม} จะได้ว่า $Q$ บนตัวเก็บประจุ $C_1$ จะเท่ากับ $Q$ บนตัวเก็บประจุ $C_2$ ดังนั้น
\[
V_\text{รวม}=V_1+V_2=\frac{Q}{C_1}+\frac{Q}{C_2}
\]
จึงได้ความจุไฟฟ้ารวมเท่ากับ
\[
\frac{1}{C_\text{รวม}}=\frac{1}{C_1}+\frac{1}{C_2}
\]
โดยเราสามารถทำแบบนี้ไปได้เรื่อย ๆ ด้วยตัวเก็บประจุกี่ตัวก็ได้ ดังนั้น
\begin{eqbox}{การต่อตัวเก็บประจุแบบอนุกรม}
    \frac{1}{C_\text{รวม}}=\frac{1}{C_1}+\frac{1}{C_2}+\dots+\frac{1}{C_n}
\end{eqbox}
และพิจารณาการต่อตัวเก็บประจุ $C_1$ และ $C_2$ แบบ\emph{ขนาน} จะได้ว่าความต่างศักย์ของตัวเก็บประจุทั้งสองจะต้องเท่ากับ (เพราะเป็นเนื้อตัวนำเดียวกัน) ดังนั้น
\[
Q_\text{รวม}=Q_1+Q_2=C_1V+C_2V
\]
จึงได้ความจุไฟฟ้ารวมเท่ากับ
\[
C_\text{รวม}=C_1+C_2
\]
โดยเช่นเดียวกับการต่อแบบอนุกรม เราสามารถทำแบบนี้ไปได้เรื่อย ๆ ด้วยตัวเก็บประจุกี่ตัวก็ได้ ดังนั้น
\begin{eqbox}{การต่อตัวเก็บประจุแบบขนาน}
    C_\text{รวม}=C_1+C_2+\dots+C_n
\end{eqbox}