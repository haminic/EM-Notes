\chapter{วงจรไฟฟ้า}
\section{การวิเคราะห์วงจร}

\subsection{กฎของ Kirchhoff}

เรามี ``กฎของ Ohm'' สำหรับแต่ละ \emph{passive component} (อุปกรณ์ที่ไม่สร้างพลังงาน) ดังนี้
\begin{lawbox}{ความสัมพันธ์ของ $V$ และ $I$ สำหรับ Passive Component}
    สำหรับดัวเก็บประจุ:
    \begin{equation}
        i = C\odv{v}{t} \label{begohm}
    \end{equation}
    สำหรับตัวต้านทาน:
    \begin{equation}
        v = iR
    \end{equation}
    และสำหรับตัวเหนี่ยวนำ:
    \begin{equation}
        v = L\odv{i}{t} \label{endohm}
    \end{equation}
\end{lawbox}
(ในที่นี้เราจะใช้ตัวอักษร $v$ และ $i$ ที่เป็นตัวพิมพ์เล็กเพื่อแทนความต่างศักย์และกระแสที่อาจขึ้นกับเวลา) ซึ่งสามารถนำมาใช้ในการวิเคราะห์วงจรได้ด้วยกฎของ Kirchhoff:

พิจารณาวงจร ณ จุด ๆ หนึ่ง ถ้าที่จุดนั้นไม่มีประจุสะสมอยู่เลยโดย \ref{contj} จะได้ว่า
\begin{ieqbox}{Kirchhoff's Current Law (KCL)}
    \sum i_\txt{in} = \sum i_\txt{out}\label{kcl}
\end{ieqbox}

โดยกฎนี้ใช้ในการวิเคราะห์วงจรแบบ\emph{โนด} (\emph{nodal analysis}) โดยเริ่มจากการตั้งศักย์ไฟฟ้าบนแต่ละโนดและกระแสที่ไหลเข้าและออกจากแต่ละโนด จากนั้นใช้ (\ref{kcl}) และ (\ref{begohm}) ถึง (\ref{endohm}) ในการเขียนทุกตัวแปรให้อยู่ในรูป $V$

ต่อมาพิจารณาวงจรที่ไม่มีการเปลี่ยนแปลงสนามแม่เหล็ก จะได้ว่า $\vb{E}$ เป็นสนามอนุรักษ์ ดังนั้น
\begin{ieqbox}{Kirchhoff's Voltage Law (KVL)}
    \sum_\txt{loop} v = 0\label{kvl}
\end{ieqbox}
หมายเหตุ: \emph{สังเกตว่าจาก KCL (\ref{kcl}) และสมบัติเชิงเส้นของ (\ref{begohm}) ถึง (\ref{endohm}) จริง ๆ แล้วความต่างศักย์นี้ไม่จำเป็นจะต้องคำนวณจากกระแสรวม แต่ขอแค่เป็นกระแสสมมติที่ครบวงปิดก็พอ}

กฎนี้ใช้ในการวิเคราะห์วงจรแบบ\emph{ลูป} (\emph{mesh analysis}) โดยเริ่มจากกำหนดกระแสที่วนอยู่ในแต่ละลูปที่กำหนดขึ้น จากนั้นใช้ (\ref{kvl}) และ (\ref{begohm}) ถึง (\ref{endohm}) ตั้งสมการตามจำนวนลูปที่กำหนดไว้เพื่อแก้หา $I$ ในแต่ละลูป

\subsection{การต่อตัวเก็บประจุ, ตัวต้านทาน, และตัวเหนี่ยวนำอย่างง่าย}

พิจารณาการต่อตัวเก็บประจุ $C_1$ และ $C_2$ แบบ\emph{อนุกรม} จะได้ว่า $q$ บนตัวเก็บประจุ $C_1$ จะเท่ากับ $q$ บนตัวเก็บประจุ $C_2$ ดังนั้น
\[ 
v_\txt{total}=v_1+v_2=\frac{q}{C_1}+\frac{q}{C_2}
\]
จึงได้ความจุไฟฟ้ารวมเท่ากับ
\[ 
\frac{1}{C_\txt{total}}=\frac{1}{C_1}+\frac{1}{C_2}
\]
โดยเราสามารถทำแบบนี้ไปได้เรื่อย ๆ ด้วยตัวเก็บประจุกี่ตัวก็ได้ ดังนั้น
\begin{eqbox}{การต่อตัวเก็บประจุแบบอนุกรม}
    \frac{1}{C_\txt{total}}=\frac{1}{C_1}+\frac{1}{C_2}+\dots+\frac{1}{C_n}
\end{eqbox}
และพิจารณาการต่อตัวเก็บประจุ $C_1$ และ $C_2$ แบบ\emph{ขนาน} จะได้ว่าความต่างศักย์ของตัวเก็บประจุทั้งสองจะต้องเท่ากัน (เพราะเป็นเนื้อตัวนำเดียวกัน) ดังนั้น
\[ 
Q_\txt{total}=Q_1+Q_2=C_1v+C_2v
\]
จึงได้ความจุไฟฟ้ารวมเท่ากับ
\[ 
C_\txt{total}=C_1+C_2
\]
โดยเช่นเดียวกับการต่อแบบอนุกรม เราสามารถทำแบบนี้ไปได้เรื่อย ๆ ด้วยตัวเก็บประจุกี่ตัวก็ได้ ดังนั้น
\newpage
\begin{eqbox}{การต่อตัวเก็บประจุแบบขนาน}
    C_\txt{total}=C_1+C_2+\dots+C_n
\end{eqbox}

ต่อมาเช่นเดียวกับตัวเก็บประจุ พิจารณาการต่อตัวต้านทาน $R_1$ และ $R_2$ แบบอนุกรม จะได้ว่ากระแสไฟฟ้าที่ไหลผ่านตัวต้านทานทั้งสองจะต้องเท่ากัน ดังนั้น
\[ 
v_\txt{total}=v_1+v_2=iR_1+iR_2=iR_1+iR_2
\]
เราสามารถทำแบบนี้ไปได้เรื่อย ๆ ด้วยตัวต้านทานกี่ตัวก็ได้ ดังนั้น
\begin{eqbox}{การต่อตัวต้านทานแบบอนุกรม}
    R_\txt{total}=R_1+R_2+\dots+R_n
\end{eqbox}
และพิจารณาการต่อตัวต้านทาน $R_1$ และ $R_2$ แบบขนาน จะได้ว่าความต่างศักย์ของตัวต้านทานทั้งสองจะต้องเท่ากัน ดังนั้น
\[ 
i_\txt{total}=i_1+i_2=\frac{v}{R_1}+\frac{v}{R_2}
\]
เราสามารถทำแบบนี้ไปได้เรื่อย ๆ ด้วยตัวต้านทานกี่ตัวก็ได้ ดังนั้น
\begin{eqbox}{การต่อตัวต้านทานแบบขนาน}
    \frac{1}{R_\txt{total}}=\frac{1}{R_1}+\frac{1}{R_2}+\dots+\frac{1}{R_n}
\end{eqbox}

สุดท้าย พิจารณาการต่อตัวเหนี่ยวนำ $L_1$ และ $L_2$ แบบอนุกรม จะได้ว่ากระแสที่ไหลผ่านตัวเหนี่ยวนำทั้งสองจะต้องเท่ากัน ดังนั้น
\[ 
v_\txt{total} = v_1 + v_2=L_1\odv{i}{t}+L_2\odv{i}{t}=L_1\odv{i}{t}+L_2\odv{i}{t}
\]
ก็จะได้
\begin{eqbox}{การต่อตัวเหนี่ยวนำแบบอนุกรม}
    L_\txt{total}=L_1+L_2+\dots+L_n
\end{eqbox}
และพิจารณาการต่อตัวเหนี่ยวนำ $L_1$ และ $L_2$ แบบขนาน จะได้ความต่างศักย์บนตัวเหนี่ยวนำทั้งสองเท่ากัน ดังนั้น
\[
\odv{i_\txt{total}}{t} = \odv{i_1}{t} + \odv{i_2}{t} = \frac{v}{L_1} + \frac{v}{L_2}
\]
ก็จะได้
\begin{eqbox}{การต่อตัวเหนี่ยวนำแบบขนาน}
    \frac{1}{L_\txt{total}}=\frac{1}{L_1}+\frac{1}{L_2}+\dots+\frac{1}{L_n}
\end{eqbox}

\section{วงจรไฟฟ้ากระแสตรง}

\subsection{อันดับของวงจร}

\begin{defbox}{อันดับของวงจร}
    \emph{อันดับของวงจร}คืออันดับของสมการเชิงอนุพันธ์ที่อธิบายวงจร เช่นวงจรไฟฟ้ากระแสตรงที่มีแค่แบตเตอรี่และตัวต้านทานไม่มีอนุพันธ์อะไรเลย จึงเป็นวงจรอันดับศูนย์
\end{defbox}

โดยวงจรอันดับหนึ่งได้แก่ วงจรที่มีตัวต้านทานและตัวเก็บประจุ (วงจร RC) และวงจรที่มีตัวต้านทานและตัวเหนี่ยวนำ (วงจร RL) และวงจรอันดับสองได้แก่วงจรที่มี passive component ทั้งสาม (วงจร RLC)

\subsection{วงจร RC}

วงจร RC เป็นวงจรอันดับหนึ่ง โดยจะยกตัวอย่างโจทย์การปล่อยประจุ (\emph{discharge}) จากตัวเก็บประจุ:

\begin{corbox}{ตัวอย่าง}
    จงหา $v(t)$ คร่อมตัวเก็บประจุของวงจรที่มีการต่อตัวเก็บประจุ $C$ และตัวต้านทาน $R$ แบบอนุกรม โดยที่ $C$ มีประจุเริ่มต้น $Q_0$
\end{corbox}

\begin{soln}
    ให้ $i_R$ และ $i_C$ คือกระแสที่ไหลออกจากจุด ๆ หนึ่งที่อยู่ฝั่งบวกของตัวเก็บประจุ จากนั้นใช้ KCL จะได้
    \begin{align*}
        i_R + i_C &= 0\\
        \frac{v}{R} + C\odv{v}{t'} &= 0\\
        -\frac{1}{RC}\odif{t'} &= \frac{1}{v} \odif{v}\\
        -\int_0^t \frac{1}{RC}\odif{t'} &= \int_{V_0}^{v(t)} \frac{1}{v} \odif{v}\\
        -\frac{1}{RC}t &= \log\ab(\frac{v(t)}{V_0})
    \end{align*}
    เนื่องจาก $V_0 = Q_0 / C$ ก็จะได้
    \[
    v(t) = \frac{Q_0}{C}\exp\ab(-\frac{1}{RC}t)
    \]
    โยเราจะเรียก $\tau \equiv RC$ ว่า\emph{ค่าคงที่เวลา} (\emph{time constant})
\end{soln}

พิจารณาวงจรที่มี $C$ ที่ \emph{steady state} (เมื่อจงจรเป็น steady current) ก็จะได้ว่า
\[
i_C = C\cancel{\odv{v}{t}} = 0 
\]
ดังนั้นเมื่อ $t\to\infty$ จะสามารถมองได้ว่า $C$ เปรียบเสมือนสายไฟขาด

\subsection{วงจร RL}

วงจร RL เป็นวงจรอันดับหนึ่ง โดยจะยกตัวอย่างโจทย์การต่อแบตเตอรี่กับวงจรที่มี $L$:

\begin{corbox}{ตัวอย่าง}
    จงหา $i(t)$ และค่าคงที่เวลา $\tau$ ของการต่อแบตเตอรี่ที่มีแรงเคลื่อนไฟฟ้า $\emf$ ในวงจรที่มีการต่อตัวต้านทาน $R$ และตัวเหนี่ยวนำ $L$ แบบอนุกรม โดยที่ ณ เวลา $t = 0$ ไม่มีกระแสไหลอยู่เลย
\end{corbox}

\begin{soln}
    วนลูปที่มีกระแส $i(t)$ รอบวงจร จากนั้นใช้ KVL จะได้
    \begin{align*}
        v_R + v_L - \emf &= 0\\
        iR + L\odv{i}{t'} &= \emf\\
        -\frac{1}{L} \odif{t'} &= \frac{1}{iR-\emf} \odif{i}\\
        -\int_0^t \frac{1}{L} \odif{t'} &= \int_0^{i(t)} \frac{1}{iR-\emf}\odif{i}\\
        -\frac{R}{L}t &= \log\ab(\frac{\emf - Ri(t)}{\emf})
    \end{align*}
    ดังนั้นก็จะได้
    \[
    i(t) = \frac{\emf}{R}\ab(1-\exp\ab(-\frac{R}{L}t))
    \]
    และค่าคงที่เวลา $\tau = L / R$
\end{soln}

พิจารณาวงจรที่มี $L$ ที่ steady state ก็จะได้ว่า
\[
v = L\cancel{\odv{i}{t}} = 0 
\]
ดังนั้นเมื่อ $t\to\infty$ จะสามารถมองได้ว่า $L$ เปรียบเสมือนสายไฟเปล่า

\subsection{วงจร RLC}

วงจร RLC เป็นวงจรอันดับสอง