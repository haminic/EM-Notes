\chapter{วงจรไฟฟ้ากระแส}
\section{ตัวเก็บประจุ, ตัวต้านทาน, ตัวเหนี่ยวนำ}
\subsection{การต่อตัวเก็บประจุอย่างง่าย}

พิจารณาการต่อตัวเก็บประจุ $C_1$ และ $C_2$ แบบ\emph{อนุกรม} จะได้ว่า $Q$ บนตัวเก็บประจุ $C_1$ จะเท่ากับ $Q$ บนตัวเก็บประจุ $C_2$ ดังนั้น
\[ 
V_\txt{total}=V_1+V_2=\frac{Q}{C_1}+\frac{Q}{C_2}
\]
จึงได้ความจุไฟฟ้ารวมเท่ากับ
\[ 
\frac{1}{C_\txt{total}}=\frac{1}{C_1}+\frac{1}{C_2}
\]
โดยเราสามารถทำแบบนี้ไปได้เรื่อย ๆ ด้วยตัวเก็บประจุกี่ตัวก็ได้ ดังนั้น
\begin{eqbox}{การต่อตัวเก็บประจุแบบอนุกรม}
    \frac{1}{C_\txt{total}}=\frac{1}{C_1}+\frac{1}{C_2}+\dots+\frac{1}{C_n}
\end{eqbox}
และพิจารณาการต่อตัวเก็บประจุ $C_1$ และ $C_2$ แบบ\emph{ขนาน} จะได้ว่าความต่างศักย์ของตัวเก็บประจุทั้งสองจะต้องเท่ากัน (เพราะเป็นเนื้อตัวนำเดียวกัน) ดังนั้น
\[ 
Q_\txt{total}=Q_1+Q_2=C_1V+C_2V
\]
จึงได้ความจุไฟฟ้ารวมเท่ากับ
\[ 
C_\txt{total}=C_1+C_2
\]
โดยเช่นเดียวกับการต่อแบบอนุกรม เราสามารถทำแบบนี้ไปได้เรื่อย ๆ ด้วยตัวเก็บประจุกี่ตัวก็ได้ ดังนั้น
\newpage
\begin{eqbox}{การต่อตัวเก็บประจุแบบขนาน}
    C_\txt{total}=C_1+C_2+\dots+C_n
\end{eqbox}

\subsection{การต่อตัวต้านทานอย่างง่าย}

พิจารณาการต่อตัวต้านทาน $R_1$ และ $R_2$ แบบอนุกรม จะได้ว่ากระแสไฟฟ้าที่ไหลผ่านตัวต้านทานทั้งสองจะต้องเท่ากัน ดังนั้น
\[ 
V_\txt{total}=V_1+V_2=IR_1+IR_2=IR_1+IR_2
\]
จึงได้ความต้านทานรวมเท่ากับ
\[ 
R_\txt{total}=R_1+R_2
\]
เราสามารถทำแบบนี้ไปได้เรื่อย ๆ ด้วยตัวต้านทานกี่ตัวก็ได้ ดังนั้น
\begin{eqbox}{การต่อตัวต้านทานแบบอนุกรม}
    R_\txt{total}=R_1+R_2+\dots+R_n
\end{eqbox}
และพิจารณาการต่อตัวต้านทาน $R_1$ และ $R_2$ แบบขนาน จะได้ว่าความต่างศักย์ของตัวต้านทานทั้งสองจะต้องเท่ากัน ดังนั้น
\[ 
I_\txt{total}=I_1+I_2=\frac{V}{R_1}+\frac{V}{R_2}
\]
จึงได้ความต้านทานรวมเท่ากับ
\[ 
\frac{1}{R_\txt{total}}=\frac{1}{R_1}+\frac{1}{R_2}
\]
เราสามารถทำแบบนี้ไปได้เรื่อย ๆ ด้วยตัวต้านทานกี่ตัวก็ได้ ดังนั้น
\begin{eqbox}{การต่อตัวต้านทานแบบขนาน}
    \frac{1}{R_\txt{total}}=\frac{1}{R_1}+\frac{1}{R_2}+\dots+\frac{1}{R_n}
\end{eqbox}

\subsection{การต่อตัวเหนี่ยวนำอย่างง่าย}

สุดท้าย พิจารณาการต่อตัวเหนี่ยวนำ $L_1$ และ $L_2$ แบบอนุกรม จะได้ว่ากระแสที่ไหลผ่านตัวเหนี่ยวนำทั้งสองจะต้องเท่ากัน ดังนั้น
\[ 
-\emf_\txt{total}=-\emf_1-\emf_2=L_1\odv{I}{t}+L_2\odv{I}{t}=L_1\odv{I}{t}+L_2\odv{I}{t}
\]
จึงได้ความเหนี่ยวนำรวมเท่ากับ
\[ 
L_\txt{total}=L_1+L_2
\]
ก็จะได้
\begin{eqbox}{การต่อตัวเหนี่ยวนำแบบอนุกรม}
    L_\txt{total}=L_1+L_2+\dots+L_n
\end{eqbox}
และพิจารณาการต่อตัวเหนี่ยวนำ $L_1$ และ $L_2$ แบบขนาน จะได้ความต่างศักย์ (ในที่นี้เท่ากับ emf) บนตัวเหนี่ยวนำทั้งสองเท่ากัน ดังนั้น
\[
\odv{I_\txt{total}}{t} = \odv{I_1}{t} + \odv{I_2}{t} = - \frac{\emf}{L_1} - \frac{\emf}{L_2}
\]
จึงได้ความเหนี่ยวนำรวมเท่ากับ
\[ 
\frac{1}{L_\txt{total}}=\frac{1}{L_1}+\frac{1}{L_2}
\]
ก็จะได้
\begin{eqbox}{การต่อตัวเหนี่ยวนำแบบขนาน}
    \frac{1}{L_\txt{total}}=\frac{1}{L_1}+\frac{1}{L_2}+\dots+\frac{1}{L_n}
\end{eqbox}
