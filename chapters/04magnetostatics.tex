\chapter{แม่เหล็กสถิต}
\section{กฎแรง Lorentz}
\subsection{แรงแม่เหล็ก}

\begin{lawbox}{แรง Lorentz}
    ประจุ $Q$ ที่เคลื่อนที่ด้วยความเร็ว $\vb{v}$ ในสนามแม่เหล็ก $\vb{B}$ จะถูกแรงแม่เหล็กกระทำดังนี้:
    \begin{equation}
        \vb{F}_\txt{mag} = Q\ab(\vb{v}\times\vb{B})
    \end{equation}
    โดยถ้ามีทั้งสนามไฟฟ้าและแม่เหล็ก:
    \begin{equation}
        \vb{F} = Q\ab\big(\vb{E} + \ab(\vb{v}\times\vb{B}))
    \end{equation}
\end{lawbox}
การเคลื่อนที่ใน $\vb{B}$ สม่ำเสมอที่น่าสนใจมีดังนี้:
\begin{enumerate}
    \item ถ้าประจุ $Q$ เคลื่อนที่ด้วยความเร็ว $\vb{v}$ ในสนาม $\vb{B}$ เพียงอย่างเดียว ส่วนของ $\vb{v}_\perp$ จะทำให้เกิดการเคลื่อนที่วงกลมตามสมการ
    \[QBR=mv=p\]
    เมื่อ $p$ คือโมเมนตัม และได้
    \[\omega = \frac{QB}{R}\]
    จะเรียกว่า\emph{ความถี่ cyclotron}
    \item ถ้าประจุ $Q$ เริ่มจากหยุดนิ่งในสนาม $\vb{E}$ และ $\vb{B}$ ที่ตั้งฉากกัน ถ้าแก้สมการมาจะได้ว่าประจุจะเคลื่อนที่เป็นรูป cycloid ที่มีรัศมี
    \[R=\frac{E}{\omega B}\]
    เมื่อ $\omega$ คือความถี่ cyclotron และศูนย์กลางวงกลมที่ทำให้เกิดรูป cycloid จะเคลื่อนที่ด้วยอัตราเร็ว
    \[u=\omega R=\frac{E}{B}\]
\end{enumerate}
ต่อมาพิจารณางานจากแรงแม่เหล็ก:
\[\odif{W}_\txt{mag}=\vb{F}_\txt{mag}\cdot\odif{\vb{l}}=Q(\vb{v}\times\vb{B})\cdot\vb{v}\odif{t}=0\]
ดังนั้นได้ว่า
\begin{corbox}{งานของแรงแม่เหล็ก}
    แรงแม่เหล็กไม่ทำงาน:
    \begin{equation}
        W_\txt{mag}=0
    \end{equation}
\end{corbox}

\subsection{กระแสไฟฟ้า}
\begin{defbox}{กระแสไฟฟ้า}
    กระแสไฟฟ้า ($\vb{I}$) ของจุดหนึ่งในสายไฟคือปริมาณประจุที่เคลื่อนที่ผ่านจุด ๆ นั้นต่อหน่วยเวลา หรือก็คือ
    \begin{equation}
        \vb{I} = \lambda\vb{v}
    \end{equation}
\end{defbox}
พิจารณา
\[\vb{F}_\txt{mag}=\int\odif{\vb{F}_\txt{mag}}=\int\ab(\vb{v}\times\vb{B})\odif{q}\]
ดังนั้นในสายไฟจะได้
\begin{eqbox}{แรงแม่เหล็กบนสายไฟ}
    \vb{F}_\txt{mag}=\int\ab(\vb{I}\times\vb{B})\odif{\ell}\label{magforcei}
\end{eqbox}
หรือก็คือ
\begin{equation}
    \vb{F}_\txt{mag}=\int\ab(I\odif{\vb{l}}\times\vb{B})
\end{equation}

ต่อมา หากประจุที่เคลื่อนที่เป็นประจุจากความหนาแน่นในสองหรือสามมิติ เราจะนิยาม:
\begin{defbox}{ความหนาแน่นกระแสไฟฟ้า}
    สำหรับประจุที่ไหลบนผิวในสองมิติ ถ้าในแถบเล็ก ๆ ที่ขนานกับทิศในการไหลของกระแส $\odif{\vb{I}}$ กว้าง $\odif{\ell_\perp}$ เราจะนิยาม\emph{ความหนาแน่นกระแสไฟฟ้าเชิงพื้นที่} ($\vb{K}$) ว่า
    \begin{equation}
        \vb{K}\equiv\frac{\odif{\vb{I}}}{\odif{\ell_\perp}}=\sigma\vb{v}
    \end{equation}
    สำหรับประจุที่ไหลในปริมาตรสามมิติ ถ้าในท่อเล็ก ๆ ที่ขนานกับทิศในการไหลของกระแส $\odif{\vb{I}}$ มีพื้นที่ $\odif{a_\perp}$ เราจะนิยาม\emph{ความหนาแน่นกระแสไฟฟ้าเชิงปริมาตร} ($\vb{J}$) ว่า
    \begin{equation}
        \vb{J}\equiv\frac{\odif{\vb{I}}}{\odif{a_\perp}}=\rho\vb{v}
    \end{equation}
\end{defbox}
และเช่นเดียวกับ (\ref{magforcei}) จะได้ว่า
\begin{corbox}{แรงแม่เหล็กบนกระแสไฟฟ้าในสองและสามมิติ}
    \begin{equation}
        \vb{F}_\txt{mag}=\int\ab(\vb{K}\times\vb{B})\odif{a}
    \end{equation}
    และ
    \begin{equation}
        \vb{F}_\txt{mag}=\int\ab(\vb{J}\times\vb{B})\odif{\tau}
    \end{equation}
\end{corbox}
