\chapter{แม่เหล็กสถิต}
\section{กฎแรง Lorentz}
\subsection{แรงแม่เหล็ก}

\begin{lawbox}{แรง Lorentz}
    ประจุ $Q$ ที่เคลื่อนที่ด้วยความเร็ว $\vb{v}$ ในสนามแม่เหล็ก $\vb{B}$ จะถูกแรงแม่เหล็กกระทำดังนี้:
    \begin{equation}
        \vb{F}_\txt{mag} = Q\ab(\vb{v}\times\vb{B})
    \end{equation}
    โดยถ้ามีทั้งสนามไฟฟ้าและแม่เหล็ก:
    \begin{equation}
        \vb{F} = Q\ab\big(\vb{E} + \ab(\vb{v}\times\vb{B}))
    \end{equation}
\end{lawbox}
การเคลื่อนที่ใน $\vb{B}$ สม่ำเสมอที่น่าสนใจมีดังนี้:
\begin{enumerate}
    \item ถ้าประจุ $Q$ เคลื่อนที่ด้วยความเร็ว $\vb{v}$ ในสนาม $\vb{B}$ เพียงอย่างเดียว ส่วนของ $\vb{v}_\perp$ จะทำให้เกิดการเคลื่อนที่วงกลมตามสมการ
    \[
    QBR=mv=p
    \]
    เมื่อ $p$ คือโมเมนตัม และได้
    \[
    \omega = \frac{QB}{R}
    \]
    จะเรียกว่า\emph{ความถี่ cyclotron}
    \item ถ้าประจุ $Q$ เริ่มจากหยุดนิ่งในสนาม $\vb{E}$ และ $\vb{B}$ ที่ตั้งฉากกัน ถ้าแก้สมการมาจะได้ว่าประจุจะเคลื่อนที่เป็นรูป cycloid ที่มีรัศมี
    \[
    R=\frac{E}{\omega B}
    \]
    เมื่อ $\omega$ คือความถี่ cyclotron และศูนย์กลางวงกลมที่ทำให้เกิดรูป cycloid จะเคลื่อนที่ด้วยอัตราเร็ว
    \[
    u=\omega R=\frac{E}{B}
    \]
\end{enumerate}
ต่อมาพิจารณางานจากแรงแม่เหล็ก:
\[
\odif{W}_\txt{mag}=\vb{F}_\txt{mag}\cdot\odif{\vb{l}}=Q(\vb{v}\times\vb{B})\cdot\vb{v}\odif{t}=0
\]
ดังนั้นได้ว่า
\begin{corbox}{งานของแรงแม่เหล็ก}
    แรงแม่เหล็กไม่ทำงาน:
    \begin{equation}
        W_\txt{mag}=0
    \end{equation}
\end{corbox}

\subsection{กระแสไฟฟ้า}
\begin{defbox}{กระแสไฟฟ้า}
    กระแสไฟฟ้า ($\vb{I}$) ของจุดหนึ่งในสายไฟคือปริมาณประจุที่เคลื่อนที่ผ่านจุด ๆ นั้นต่อหน่วยเวลา หรือก็คือ
    \begin{equation}
        \vb{I} = \lambda\vb{v}
    \end{equation}
\end{defbox}
พิจารณา
\[
\vb{F}_\txt{mag}=\int\odif{\vb{F}_\txt{mag}}=\int\ab(\vb{v}\times\vb{B})\odif{q}
\]
ดังนั้นในสายไฟจะได้
\begin{eqbox}{แรงแม่เหล็กบนสายไฟ}
    \vb{F}_\txt{mag}=\int\ab(\vb{I}\times\vb{B})\odif{\ell}\label{magforcei}
\end{eqbox}
หรือก็คือ
\begin{equation}
    \vb{F}_\txt{mag}=\int\ab(I\odif{\vb{l}}\times\vb{B})
\end{equation}

ต่อมา หากประจุที่เคลื่อนที่เป็นประจุจากความหนาแน่นในสองหรือสามมิติ เราจะนิยาม:
\begin{defbox}{ความหนาแน่นกระแสไฟฟ้า}
    สำหรับประจุที่ไหลบนผิวในสองมิติ ถ้าในแถบเล็ก ๆ ที่ขนานกับทิศในการไหลของกระแส $\odif{\vb{I}}$ กว้าง $\odif{\ell_\perp}$ เราจะนิยาม\emph{ความหนาแน่นกระแสไฟฟ้าเชิงพื้นที่} ($\vb{K}$) ว่า
    \begin{equation}
        \vb{K}\equiv\frac{\odif{\vb{I}}}{\odif{\ell_\perp}}=\sigma\vb{v}
    \end{equation}
    สำหรับประจุที่ไหลในปริมาตรสามมิติ ถ้าในท่อเล็ก ๆ ที่ขนานกับทิศในการไหลของกระแส $\odif{\vb{I}}$ มีพื้นที่ $\odif{a_\perp}$ เราจะนิยาม\emph{ความหนาแน่นกระแสไฟฟ้าเชิงปริมาตร} ($\vb{J}$) ว่า
    \begin{equation}
        \vb{J}\equiv\frac{\odif{\vb{I}}}{\odif{a_\perp}}=\rho\vb{v}\label{curdensdef}
    \end{equation}
\end{defbox}
และเช่นเดียวกับ (\ref{magforcei}) จะได้ว่า
\begin{corbox}{แรงแม่เหล็กบนกระแสในสองและสามมิติ}
    \begin{equation}
        \vb{F}_\txt{mag}=\int\ab(\vb{K}\times\vb{B})\odif{a}
    \end{equation}
    และ
    \begin{equation}
        \vb{F}_\txt{mag}=\int\ab(\vb{J}\times\vb{B})\odif{\tau}
    \end{equation}
\end{corbox}

จากสมการ (\ref{curdensdef}) จะได้ว่า
\begin{equation}
    I=\int_{\sur}\vb{J}\cdot \odif{\vb{a}}
\end{equation}
และเนื่องจากประจุที่ไหลออก ($I$) จะต้องเท่ากับประจุที่หายไป ดังนั้น
\[
\int_\vol\ab(\gd\cdot \vb{J}) \odif{\tau} = \oint_{\del\vol} \vb{J}\cdot \odif{\vb{a}} = I = -\odv{Q_\txt{enc}}{t} = -\odv{}{t}\int_\vol\rho \odif{\tau} = -\int_\vol\ab(\pdv{\rho}{\tau})\odif{\tau}
\]
ก็จะได้ว่า
\begin{ieqbox}{สมการความต่อเนื่อง}
    \gd\cdot \vb{J} = -\pdv{\rho}{t}\label{contj}
\end{ieqbox}

\section{กฏของ Biot-Savart}
\subsection{ระบบกระแสคงที่}
ในบทก่อน ๆ เราได้หาสนามไฟฟ้าในระบบที่เป็นประจุหยุดหนึ่งไปแล้วหรือก็คือเป็นระบบ\emph{ไฟฟ้าสถิต} (\emph{electrostatics}) ต่อมาในกรณีสนามแม่เหล็ก ในการที่ระบบจะเป็น\emph{แม่เหล็กสถิต} (\emph{magnetostatics}) ระบบจะต้องมีกระแสคงเดิมตลอดเวลา หรือก็คือ:
\begin{equation}
    \pdv{\rho}{t} = 0\qq{และ}\pdv{\vb{J}}{t} = 0
\end{equation}
เมื่อนำไปแทนใน \ref{contj} จะได้ว่า
\begin{equation}
    \gd\cdot\vb{J}=0
\end{equation}
โดยในระบบกระแสคงที่นี้เราจะหาสนามแม่เหล็กได้จาก:
\begin{lawbox}{กฎของ Biot-Savart}
    สนามแม่เหล็ก $\vb{B}$ ที่ตำแหน่ง $\vb{r}$ ในระบบที่เป็นแม่เหล็กสถิต หาได้จาก
    \begin{equation}
        \vb{B}(\vb{r}) = \frac{\mu_0}{4\pi}\int\frac{\vb{I}\times\vus{\rad}}{\rad^2}\odif{\ell'} = \frac{\mu_0}{4\pi}I\int\frac{\odif{\vbs{\ell}'}\times\vus{\rad}}{\rad^2}\label{biot}
    \end{equation}
\end{lawbox}
เมื่อ $\mu_0$ คือ\emph{สภาพซึมผ่านได้ของสุญญากาศ} (\emph{permeability of free space}) โดยในกรณีความหนาแน่นกระแส:
\begin{eqbox}{กฎของ Biot-Savart ของกระแสในสองและสามมิติ}
    \vb{B}(\vb{r}) = \frac{\mu_0}{4\pi}\int\frac{\vb{K}(\vb{r}')\times\vus{\rad}}{\rad^2}\odif{a'}\qq{และ}\vb{B}(\vb{r}) = \frac{\mu_0}{4\pi}\int\frac{\vb{J}(\vb{r}')\times\vus{\rad}}{\rad^2}\odif{\tau'}\label{biotdim}
\end{eqbox}

\section{Divergence และ Curl ของสนามแม่เหล็กสถิต}
\subsection{Divergence ของสนามแม่เหล็กสถิต}
พิจารณา (\ref{biotdim}) จะได้ว่า
\begin{align*}
    \gd\cdot\vb{B} &= \frac{\mu_0}{4\pi}\int\gd\cdot\ab(\vb{J}\times\frac{\vus{\rad}}{\rad^2})\odif{\tau'}\\
    &= \frac{\mu_0}{4\pi}\int\ab(\frac{\vus{\rad}}{\rad^2}\cdot\ab(\gd\times\vb{J})-\vb{J}\cdot\ab(\gd\times\frac{\vus{\rad}}{\rad^2})\odif{\tau'})
\end{align*}
เนื่องจาก $\vb{J}$ อยู่ในพิกัด $(x',y',z')$ จึงได้ว่า $\gd\times\vb{J}=\vb{0}$ และจาก
\[
\gd\times\frac{\vus{\rad}}{\rad^2} = \vb{0}
\]
ดังนั้น
\begin{ieqbox}{กฎของ Gauss สำหรับสนามแม่เหล็ก}
    \gd\cdot\vb{B} = 0
\end{ieqbox}
\subsection{Curl ของสนามแม่เหล็กสถิต}
เช่นเดิม จาก (\ref{biotdim}) จะได้ว่า
\begin{align*}
    \gd\times\vb{B} &= \frac{\mu_0}{4\pi} \int\gd\times\ab(\vb{J}\times\frac{\vus{\rad}}{\rad^2})\odif{\tau'}\\
    &= \frac{\mu_0}{4\pi}\int\ab(\ab(\frac{\vus{\rad}}{\rad^2}\cdot\gd)\,\vb{J} - \ab(\vb{J}\cdot\gd)\frac{\vus{\rad}}{\rad^2} + \vb{J}\,\ab(\gd\cdot\frac{\vus{\rad}}{\rad^2}) - \frac{\vus{\rad}}{\rad^2}\ab(\gd\cdot\vb{J}))\odif{\tau'}
\end{align*}
เนื่องจาก $\vb{J}$ ขึ้นกับพิกัด $(x',y',z')$:
\begin{equation}
    \gd\times\vb{B} = \frac{\mu_0}{4\pi} \int\ab(\vb{J}\,\ab(\gd\cdot\frac{\vus{\rad}}{\rad^2}) - \ab(\vb{J}\cdot\gd)\frac{\vus{\rad}}{\rad^2}) \odif{\tau'}\label{proofcurlb}\tag{$\star$1}
\end{equation}
พิจารณาพจน์ด้านหลัง เนื่องจาก $\vus{\rad} / \rad^2 = f(\vb{r} - \vb{r}')$ ดังนั้น $\gd = -\gd'$ จะได้ว่า
\[
\int -\ab(\vb{J}\cdot\gd)\frac{\vus{\rad}}{\rad^2} \odif{\tau'} = \int \ab(\vb{J}\cdot\gd')\frac{\vus{\rad}}{\rad^2} \odif{\tau'}
\]
คิดแยกแกน โดยให้ $\vol$ คือปริมาตรที่อินทิเกรต (ปริมาตรที่ใหญ่มาก ๆ):
\begin{align*}
    \ab(\int_\vol -\ab(\vb{J}\cdot\gd)\frac{\vus{\rad}}{\rad^2} \odif{\tau'})_x &= \int_\vol \ab(\vb{J}\cdot\gd')\,\frac{x-x'}{\rad^2} \odif{\tau'}\\
    &= \int_\vol \gd'\cdot\ab(\frac{x-x'}{\rad^2}\vb{J})\odif{\tau'} - \cancel{\int_\vol \ab(\gd'\cdot\vb{J})\,\frac{x-x'}{\rad^2} \odif{\tau'}}\\
    &= \oint_{\del\vol} \ab(\frac{x-x'}{\rad^2}\vb{J})\cdot\odif{\vb{a}}
\end{align*}
เนื่องจาก $\vol$ ใหญ่มาก ดังนั้นจะได้ว่าไม่มีกระแสไหลออกจากระบบเลย ดังนั้น
\[
\int_\vol -\ab(\vb{J}\cdot\gd)\frac{\vus{\rad}}{\rad^2} \odif{\tau'} = \vb{0}
\]
นำไปแทนใน (\ref{proofcurlb}) จะได้ว่า
\begin{align*}
    \gd\times\vb{B} &= \frac{\mu_0}{4\pi}\int\vb{J}\,\ab(\gd\cdot\frac{\vus{\rad}}{\rad^2})\odif{\tau'}\\
    &= \frac{\mu_0}{4\pi} \int \vb{J}\,\ab(4\pi\,\delta^3(\vbs{\rad})) \odif{\tau'}
\end{align*}
ดังนั้น
\begin{ieqbox}{กฎของ Ampère (Differential Form)}
    \gd\times\vb{B} = \mu_0\vb{J}
\end{ieqbox}