\chapter{แม่เหล็กสถิต}
\section{กฎแรง Lorentz}
\subsection{แรงแม่เหล็ก}

\begin{lawbox}{แรง Lorentz}
    ประจุ $Q$ ที่เคลื่อนที่ด้วยความเร็ว $\vb{v}$ ในสนามแม่เหล็ก $\vb{B}$ จะถูกแรงแม่เหล็กกระทำดังนี้:
    \begin{equation}
        \vb{F}_\txt{mag} = Q\ab(\vb{v}\times\vb{B})
    \end{equation}
    โดยถ้ามีทั้งสนามไฟฟ้าและแม่เหล็ก:
    \begin{equation}
        \vb{F} = Q\ab\big(\vb{E} + \ab(\vb{v}\times\vb{B}))
    \end{equation}
\end{lawbox}
การเคลื่อนที่ใน $\vb{B}$ สม่ำเสมอที่น่าสนใจมีดังนี้:
\begin{enumerate}
    \item ถ้าประจุ $Q$ เคลื่อนที่ด้วยความเร็ว $\vb{v}$ ในสนาม $\vb{B}$ เพียงอย่างเดียว ส่วนของ $\vb{v}_\perp$ จะทำให้เกิดการเคลื่อนที่วงกลมตามสมการ
    \[
    QBR=mv=p
    \]
    เมื่อ $p$ คือโมเมนตัม และได้
    \[
    \omega = \frac{QB}{R}
    \]
    จะเรียกว่า\emph{ความถี่ cyclotron}
    \item ถ้าประจุ $Q$ เริ่มจากหยุดนิ่งในสนาม $\vb{E}$ และ $\vb{B}$ ที่ตั้งฉากกัน ถ้าแก้สมการมาจะได้ว่าประจุจะเคลื่อนที่เป็นรูป cycloid ที่มีรัศมี
    \[
    R=\frac{E}{\omega B}
    \]
    เมื่อ $\omega$ คือความถี่ cyclotron และศูนย์กลางวงกลมที่ทำให้เกิดรูป cycloid จะเคลื่อนที่ด้วยอัตราเร็ว
    \[
    u=\omega R=\frac{E}{B}
    \]
\end{enumerate}
ต่อมาพิจารณางานจากแรงแม่เหล็ก:
\[
\odif{W}_\txt{mag}=\vb{F}_\txt{mag}\cdot\odif{\vb{l}}=Q(\vb{v}\times\vb{B})\cdot\vb{v}\odif{t}=0
\]
ดังนั้นได้ว่า
\begin{corbox}{งานของแรงแม่เหล็ก}
    แรงแม่เหล็กไม่ทำงาน:
    \begin{equation}
        W_\txt{mag}=0
    \end{equation}
\end{corbox}

\subsection{กระแสไฟฟ้า}
\begin{defbox}{กระแสไฟฟ้า}
    กระแสไฟฟ้า ($\vb{I}$) ของจุดหนึ่งในสายไฟคือปริมาณประจุที่เคลื่อนที่ผ่านจุด ๆ นั้นต่อหน่วยเวลา หรือก็คือ
    \begin{equation}
        \vb{I} = \lambda\vb{v}
    \end{equation}
\end{defbox}
พิจารณา
\[
\vb{F}_\txt{mag}=\int\odif{\vb{F}_\txt{mag}}=\int\ab(\vb{v}\times\vb{B})\odif{q}
\]
ดังนั้นในสายไฟจะได้
\begin{eqbox}{แรงแม่เหล็กบนสายไฟ}
    \vb{F}_\txt{mag}=\int\ab(\vb{I}\times\vb{B})\odif{\ell}\label{magforcei}
\end{eqbox}
หรือก็คือ
\begin{equation}
    \vb{F}_\txt{mag}=\int\ab(I\odif{\vb{l}}\times\vb{B})
\end{equation}

ต่อมา หากประจุที่เคลื่อนที่เป็นประจุจากความหนาแน่นในสองหรือสามมิติ เราจะนิยาม:
\begin{defbox}{ความหนาแน่นกระแสไฟฟ้า}
    สำหรับประจุที่ไหลบนผิวในสองมิติ ถ้าในแถบเล็ก ๆ ที่ขนานกับทิศในการไหลของกระแส $\odif{\vb{I}}$ กว้าง $\odif{\ell_\perp}$ เราจะนิยาม\emph{ความหนาแน่นกระแสไฟฟ้าเชิงพื้นที่} ($\vb{K}$) ว่า
    \begin{equation}
        \vb{K}\equiv\frac{\odif{\vb{I}}}{\odif{\ell_\perp}}=\sigma\vb{v}
    \end{equation}
    สำหรับประจุที่ไหลในปริมาตรสามมิติ ถ้าในท่อเล็ก ๆ ที่ขนานกับทิศในการไหลของกระแส $\odif{\vb{I}}$ มีพื้นที่ $\odif{a_\perp}$ เราจะนิยาม\emph{ความหนาแน่นกระแสไฟฟ้าเชิงปริมาตร} ($\vb{J}$) ว่า
    \begin{equation}
        \vb{J}\equiv\frac{\odif{\vb{I}}}{\odif{a_\perp}}=\rho\vb{v}\label{curdensdef}
    \end{equation}
\end{defbox}
โดยเราจึงสามารถหากระแสไฟฟ้าที่ไหลผ่านผิว ๆ หนึ่งหรือเส้น ๆ หนึ่งได้จาก
\begin{equation}
    I = \int \vb{K}\cdot\odif{\vbs{\ell}}\qq{และ} I = \int \vb{J}\cdot\odif{\vb{a}}
\end{equation}
และเช่นเดียวกับ (\ref{magforcei}) จะได้ว่า
\begin{corbox}{แรงแม่เหล็กบนกระแสในสองและสามมิติ}
    \begin{equation}
        \vb{F}_\txt{mag}=\int\ab(\vb{K}\times\vb{B})\odif{a}
    \end{equation}
    และ
    \begin{equation}
        \vb{F}_\txt{mag}=\int\ab(\vb{J}\times\vb{B})\odif{\tau}
    \end{equation}
\end{corbox}

จากสมการ (\ref{curdensdef}) จะได้ว่า
\begin{equation}
    I=\int_{\sur}\vb{J}\cdot \odif{\vb{a}}
\end{equation}
และเนื่องจากประจุที่ไหลออก ($I$) จะต้องเท่ากับประจุที่หายไป ดังนั้น
\[
\int_\vol\ab(\gd\cdot \vb{J}) \odif{\tau} = \oint_{\del\vol} \vb{J}\cdot \odif{\vb{a}} = I = -\odv{Q_\txt{enc}}{t} = -\odv{}{t}\int_\vol\rho \odif{\tau} = -\int_\vol\ab(\pdv{\rho}{\tau})\odif{\tau}
\]
ก็จะได้ว่า
\begin{ieqbox}{สมการความต่อเนื่อง}
    \gd\cdot \vb{J} = -\pdv{\rho}{t}\label{contj}
\end{ieqbox}

\section{กฎของ Biot-Savart}
\subsection{ระบบกระแสคงที่}
ในบทก่อน ๆ เราได้หาสนามไฟฟ้าในระบบที่เป็นประจุหยุดนิ่งไปแล้วหรือก็คือเป็นระบบ\emph{ไฟฟ้าสถิต} (\emph{electrostatics}) ต่อมาในกรณีสนามแม่เหล็ก ในการที่ระบบจะเป็น\emph{แม่เหล็กสถิต} (\emph{magnetostatics}) ระบบจะต้องมีกระแสคงเดิมตลอดเวลา หรือก็คือ:
\begin{equation}
    \pdv{\rho}{t} = 0\qq{และ}\pdv{\vb{J}}{t} = 0
\end{equation}
เมื่อนำไปแทนใน (\ref{contj}) จะได้ว่า
\begin{equation}
    \gd\cdot\vb{J}=0\label{gdjzero}
\end{equation}
โดยในระบบกระแสคงที่นี้เราจะหาสนามแม่เหล็กได้จาก:
\begin{lawbox}{กฎของ Biot-Savart}
    สนามแม่เหล็ก $\vb{B}$ ที่ตำแหน่ง $\vb{r}$ ในระบบที่เป็นแม่เหล็กสถิต หาได้จาก
    \begin{equation}
        \vb{B}(\vb{r}) = \frac{\mu_0}{4\pi}\int\frac{\vb{I}\times\vus{\rad}}{\rad^2}\odif{\ell'} = \frac{\mu_0}{4\pi}I\int\frac{\odif{\vbs{\ell}'}\times\vus{\rad}}{\rad^2}\label{biot}
    \end{equation}
\end{lawbox}
เมื่อ $\mu_0$ คือ\emph{สภาพซึมผ่านได้ของสุญญากาศ} (\emph{permeability of free space}) โดยในกรณีความหนาแน่นกระแส:
\begin{eqbox}{กฎของ Biot-Savart ของกระแสในสองและสามมิติ}
    \vb{B}(\vb{r}) = \frac{\mu_0}{4\pi}\int\frac{\vb{K}(\vb{r}')\times\vus{\rad}}{\rad^2}\odif{a'}\qq{และ}\vb{B}(\vb{r}) = \frac{\mu_0}{4\pi}\int\frac{\vb{J}(\vb{r}')\times\vus{\rad}}{\rad^2}\odif{\tau'}\label{biotdim}
\end{eqbox}

\section{Divergence และ Curl ของสนามแม่เหล็กสถิต}
\subsection{Divergence ของสนามแม่เหล็กสถิต}
พิจารณา (\ref{biotdim}) จะได้ว่า
\begin{align*}
    \gd\cdot\vb{B} &= \frac{\mu_0}{4\pi}\int\gd\cdot\ab(\vb{J}\times\frac{\vus{\rad}}{\rad^2})\odif{\tau'}\\
    &= \frac{\mu_0}{4\pi}\int\ab(\frac{\vus{\rad}}{\rad^2}\cdot\ab(\gd\times\vb{J})-\vb{J}\cdot\ab(\gd\times\frac{\vus{\rad}}{\rad^2})\odif{\tau'})
\end{align*}
เนื่องจาก $\vb{J}$ อยู่ในพิกัด $(x',y',z')$ จึงได้ว่า $\gd\times\vb{J}=\vb{0}$ และจาก
\[
\gd\times\frac{\vus{\rad}}{\rad^2} = \vb{0}
\]
ดังนั้น
\begin{ieqbox}{กฎของ Gauss สำหรับสนามแม่เหล็ก}
    \gd\cdot\vb{B} = 0\label{gaussb}
\end{ieqbox}
\subsection{Curl ของสนามแม่เหล็กสถิต}
เช่นเดิม จาก (\ref{biotdim}) จะได้ว่า
\begin{align*}
    \gd\times\vb{B} &= \frac{\mu_0}{4\pi} \int\gd\times\ab(\vb{J}\times\frac{\vus{\rad}}{\rad^2})\odif{\tau'}\\
    &= \frac{\mu_0}{4\pi}\int\ab(\ab(\frac{\vus{\rad}}{\rad^2}\cdot\gd)\,\vb{J} - \ab(\vb{J}\cdot\gd)\frac{\vus{\rad}}{\rad^2} + \vb{J}\,\ab(\gd\cdot\frac{\vus{\rad}}{\rad^2}) - \frac{\vus{\rad}}{\rad^2}\ab(\gd\cdot\vb{J}))\odif{\tau'}
\end{align*}
เนื่องจาก $\vb{J}$ ขึ้นกับพิกัด $(x',y',z')$:
\begin{equation}
    \gd\times\vb{B} = \frac{\mu_0}{4\pi} \int\ab(\vb{J}\,\ab(\gd\cdot\frac{\vus{\rad}}{\rad^2}) - \ab(\vb{J}\cdot\gd)\frac{\vus{\rad}}{\rad^2}) \odif{\tau'}\label{proofcurlb}\tag{$\dagger$}
\end{equation}
พิจารณาพจน์ด้านหลัง เนื่องจาก $\vus{\rad} / \rad^2 = f(\vb{r} - \vb{r}')$ ดังนั้น $\gd = -\gd'$ จะได้ว่า
\[
\int -\ab(\vb{J}\cdot\gd)\frac{\vus{\rad}}{\rad^2} \odif{\tau'} = \int \ab(\vb{J}\cdot\gd')\frac{\vus{\rad}}{\rad^2} \odif{\tau'}
\]
คิดแยกแกน โดยให้ $\vol$ คือปริมาตรที่อินทิเกรต (ปริมาตรที่ใหญ่มาก ๆ):
\begin{align*}
    \ab(\int_\vol -\ab(\vb{J}\cdot\gd)\frac{\vus{\rad}}{\rad^2} \odif{\tau'})_x &= \int_\vol \ab(\vb{J}\cdot\gd')\,\frac{x-x'}{\rad^2} \odif{\tau'}\\
    &= \int_\vol \gd'\cdot\ab(\frac{x-x'}{\rad^2}\vb{J})\odif{\tau'} - \cancel{\int_\vol \ab(\gd'\cdot\vb{J})\,\frac{x-x'}{\rad^2} \odif{\tau'}}\\
    &= \oint_{\del\vol} \ab(\frac{x-x'}{\rad^2}\vb{J})\cdot\odif{\vb{a}}
\end{align*}
เนื่องจาก $\vol$ ใหญ่มาก ดังนั้นจะได้ว่าไม่มีกระแสไหลออกจากระบบเลย ดังนั้น
\[
\int_\vol -\ab(\vb{J}\cdot\gd)\frac{\vus{\rad}}{\rad^2} \odif{\tau'} = \vb{0}
\]
นำไปแทนใน (\ref{proofcurlb}) จะได้ว่า
\begin{align*}
    \gd\times\vb{B} &= \frac{\mu_0}{4\pi}\int\vb{J}\,\ab(\gd\cdot\frac{\vus{\rad}}{\rad^2})\odif{\tau'}\\
    &= \frac{\mu_0}{4\pi} \int \vb{J}\,\ab(4\pi\,\delta^3(\vbs{\rad})) \odif{\tau'}
\end{align*}
ดังนั้น
\begin{ieqbox}{กฎของ Ampère (Differential Form)}
    \gd\times\vb{B} = \mu_0\vb{J}\label{amperedif}
\end{ieqbox}
ต่อมาอินทิเกรตบนผิวใด ๆ:
\[
\int \ab(\gd\times\vb{B})\cdot\odif{\vb{a}} = \mu_0 \int \vb{J}\cdot\odif{\vb{a}} = \mu_0 I_\txt{enc}
\]
โดย Stokes' Theorem จะได้ว่า
\begin{ieqbox}{กฎของ Ampère (Integral Form)}
    \oint\vb{B}\cdot\odif{\vbs{\ell}} = \mu_0 I_\txt{enc}\label{ampereint}
\end{ieqbox}

\section{เวกเตอร์ศักย์แม่เหล็ก}
\subsection{นิยามศักย์แม่เหล็ก}
ในบทสนามไฟฟ้า เนื่องจาก $\gd\times\vb{E} = \vb{0}$ ทำให้เราสามารถนิยามสนามสเกลาร์ $V$ (ศักย์ไฟฟ้า) ซึ่ง
\[
\vb{E} = -\gd V
\]
ในกรณีของสนามแม่เหล็ก เรามี (\ref{gaussb}) ที่กล่าวว่า $\gd\cdot\vb{B} = 0$ จึงทำให้เราสามารถนิยาม\emph{ศักย์แม่เหล็ก}ซึ่งเป็นสนามเวกเตอร์ได้ว่า
\begin{defbox}{ศักย์แม่เหล็ก}
    \begin{equation}
        \vb{B} = \gd\times\vb{A}\label{magpot}
    \end{equation}
\end{defbox}
พิจารณา (\ref{amperedif}):
\begin{align*}
    \mu_0\vb{J} &= \gd\times\vb{B}\\
    &= \gd\times\ab(\gd\times\vb{A})\\
    &= \gd(\gd\cdot\vb{A}) - \nabla^2\vb{A}
\end{align*}
สังเกตว่าถ้า $\vb{A}_0$ สอดคล้องกับ (\ref{magpot}) แล้ว $\vb{A} = \vb{A}_0 + \gd\lambda$ ก็สอดคล้องด้วย พิจารณา $\gd\cdot\vb{A}$:
\begin{align*}
    \gd\cdot\vb{A} = \gd\cdot\vb{A}_0 + \nabla^2\lambda
\end{align*}
ดังนั้นเราจึงสามารถเลือกสนามศักย์แม่เหล็กให้มี $\gd\cdot\vb{A} = 0$ ได้เสมอ (เพราะเรารู้ว่าสมการ Poisson $\nabla^2\lambda = -f(x) = -\gd\cdot\vb{A}_0$ มีคำตอบ) ก็จะได้ว่า
\begin{eqbox}{สมการ Poisson ของศักย์แม่เหล็ก}
    \nabla^2 \vb{A} = \mu_0\vb{J}\label{poissonb}
\end{eqbox}
ในบทไฟฟ้าสถิตเรามีคำตอบของสมการ $\nabla^2 V = \rho / \eps_0$ อยู่แล้ว (เมื่อ $\rho\to 0$ เมื่อ $\vb{r}\to\infty$) คือ
\[
V(\vb{r}) = \frac{1}{4\pi\eps_0}\int\frac{\rho(\vb{r}')}{\rad} \odif{\tau'}
\]
จังดัดแปลงให้เป็นคำตอบของ (\ref{poissonb}) ได้ว่า
\begin{ieqbox}{ศักย์แม่เหล็กจากสนามกระแส}
    \vb{A}(\vb{r}) = \frac{\mu_0}{4\pi}\int\frac{\vb{J}(\vb{r}')}{\rad} \odif{\tau'}
\end{ieqbox}
หรือสำหรับหนึ่งและสองมิติ:
\begin{eqbox}{ศักย์แม่เหล็กจากสนามกระแสในหนึ่งและสองมิติ}
    \vb{A}(\vb{r}) = \frac{\mu_0}{4\pi} \int \frac{\vb{K}(\vb{r}')}{\rad} \odif{a'}\qq{และ}\vb{A}(\vb{r}) = \frac{\mu_0}{4\pi} \int \frac{\vb{I}(\vb{r})}{\rad} \odif{\ell'} = \frac{\mu_0 I}{4\pi} \int \frac{1}{\rad} \odif{\vbs{\ell}'}\label{magpotfromik}
\end{eqbox}
โดยสมการศักย์นี้ใช้ได้เมื่อ $\vb{J}\to\vb{0}$ เมื่อ $\vb{r}\to\infty$ แต่ถ้าไม่ใช่ อาจต้องหาศักย์โดยใช้วิธีอื่น วิธีหนึ่งคือสังเกตว่า
\[
\oint_{\del\sur} \vb{A}\cdot\odif{\vbs{\ell}} = \int_\sur \ab(\gd\times\vb{A})\cdot\odif{\vb{a}} = \int_\sur\vb{B}\cdot\odif{\vb{a}} 
\]
โดยเราจะเรียกพจน์ฝั่งขวาว่า\emph{ฟลักซ์แม่เหล็ก} ($\Phi_B$):
\begin{defbox}{ฟลักซ์แม่เหล็ก}
    ฟลักซ์ของ $\vb{B}$ ที่ผ่านผิว $\sur$ คือ
    \begin{equation}
        \Phi_B\equiv\int_\sur\vb{B}\cdot\odif{\vb{a}}
    \end{equation}
\end{defbox}
ดังนั้นสมการด้านบนก็จะได้ว่า
\begin{equation}
    \oint\vb{A}\cdot\odif{\vbs{\ell}} = \Phi_B\label{atofluxb}
\end{equation}
ซึ่งสามารถนำมาใช้หา $\vb{A}$ ได้เช่นเดียวกับการใช้ (\ref{ampereint}) ในการหา $\vb{B}$ ในระบบที่มีความสมมาตร

\subsection{สภาวะขอบเขต}
ต่อมาเรามาหาสภาวะขอบเขตของสนามแม่เหล็ก $\vb{B}$ และศักย์แม่เหล็ก $\vb{A}$ เช่นเดียวกับในบทไฟฟ้าสถิต โดยพิจารณาที่บริเวณแผ่นที่มีกระแส $\vb{K}$ ไหลอยู่:
\begin{enumerate}
    \item พิจารณาผิว Gaussian ทรงกระบอกที่บางมาก ๆ ดั่งในตอนหาสภาวะขอบเขตของ $\vb{E}$ และใช้ (\ref{gaussb}) จะได้ว่า
    \[
    B_\txt{above}^\perp - B_\txt{below}^\perp = 0
    \]
    ดังนั้นส่วนของ $\vb{B}$ ที่ตั้งฉากกับแผ่นกระแสจะต่อเนื่อง
    \item พิจารณาลูป Amperian รูปสี่เหลี่ยมผืนผ้าแคบ ๆ ที่มีด้านขนานกับแผ่นกระแสแต่ตั้งฉากกับ $\vb{K}$ จะได้ว่า
    \[
    B_\txt{above}^\| - B_\txt{below}^\| = \mu_0K
    \]
    ดังนั้นส่วนของ $\vb{B}$ ที่ขนานกับแผ่นกระแสแต่ตั้งฉากกับ $\vb{K}$ จะไม่ต่อเนื่องแบบกระโดดด้วยผลต่าง $\mu_0K$
    \item เนื่องจาก $\gd\cdot\vb{A} = 0$ จะได้ว่า $A^\perp$ ต่อเนื่อง และจาก (\ref{atofluxb}) ก็จะได้ว่า $A^\|$ ก็ต่อเนื่อง ดังนั้น
    \[
    \vb{A}_\txt{above} = \vb{A}_\txt{below}
    \]
    หรือก็คือ $\vb{A}$ ต่อเนื่องเมื่อผ่านแนวแผ่นกระแส
    \item กำหนดให้ $\Delta\vb{A} \equiv \vb{A}_\txt{above}-\vb{A}_\txt{below}$, $\vu{k}\equiv\vu{K}$, และ $\vu{p}\equiv\vu{k}\times\vu{n}$ จากข้อ 1. และ 2. จะได้ว่า 
    \begin{align*}
        \mu_0\,\ab(\vb{K}\times\vu{n}) &= \vb{B}_\txt{above}-\vb{B}_\txt{below}\\
        \mu_0K\vu{p} &= \gd\times\ab(\vb{A}_\txt{above}-\vb{A}_\txt{below})\\
        &=\gd\times\Delta\vb{A}\\
        &=  \begin{vmatrix}
                \vu{k}      & \vu{n}    & \vu{p}\\
                D_k         & D_n       & D_p\\
                \Delta A_k  & \Delta A_n& \Delta A_p
            \end{vmatrix}\\
        &= \vu{p}\ab(\cancel{\pdv{}{k}\Delta A_n} - \pdv{}{n}\Delta A_k) + \vu{k}\ab(\pdv{}{n}\Delta A_p - \cancel{\pdv{}{p}\Delta A_n})
    \end{align*}
    (เนื่องจาก $\vb{A}$ ต่อเนื่อง $\del/\del k$ และ $\del/\del p$ ของข้างบนและข้างล่างจึงเท่ากัน) ดังนั้นก็จะได้ว่า
    \begin{equation}
        \pdv{}{n}\Delta A_k = -\mu_0 K\qq{และ}\pdv{}{n}\Delta A_p = 0\tag{$\heartsuit$1}\label{proofbounda1}
    \end{equation}
    ต่อมา จาก
    \[
    0=\gd\cdot\Delta\vb{A} = \cancel{\pdv{}{k}\Delta A_k} + \pdv{}{n}\Delta A_n + \cancel{\pdv{}{p}\Delta A_p}
    \]
    ก็จะได้ว่า
    \begin{equation}
        \pdv{}{n}\Delta A_n = 0\tag{$\heartsuit$2}\label{proofbounda2}
    \end{equation}
    จาก (\ref{proofbounda1}) และ (\ref{proofbounda2}) ก็จะได้
    \[
    \pdv{\vb{A}_\txt{above}}{n}-\pdv{\vb{A}_\txt{below}}{n}=-\mu_0\vb{K}
    \]
\end{enumerate}
สรุปก็คือ
\begin{lawbox}{สภาวะขอบเขตของ $\vb{B}$ และ $\vb{A}$ เมื่อผ่านแผ่นกระแส}
    บนแผ่นประจุที่มีความหนาแน่นเชิงพื้นที่ $\sigma$ จะได้ว่า
    \begin{equation}
        \vb{A}_\txt{above}=\vb{A}_\txt{below}
    \end{equation}
    และ
    \begin{equation}
        \vb{B}_\txt{above}-\vb{B}_\txt{below}=\mu_0\,\ab(\vb{K}\times\vu{n})
    \end{equation}
    เมื่อ $\vu{n}$ คือเวกเตอร์หนึ่งหน่วยที่ตั้งฉากกับแผ่นกระแสที่ชี้จากด้านล่างไปด้านบน หรือก็จะได้
    \begin{equation}
        \pdv{\vb{A}_\txt{above}}{n}-\pdv{\vb{A}_\txt{below}}{n}=-\mu_0\vb{K}
    \end{equation}
\end{lawbox}

\subsection{การกระจาย Multipole ของศักย์แม่เหล็ก}
พิจารณาการกระจาย multipole ของ $\vb{A}$ (อนุกรมกำลังในรูป $1 / r$) โดยใช้ (\ref{mpradinv}) และ (\ref{magpotfromik}):
\begin{align*}
    \vb{A}(\vb{r}) &= \frac{\mu_0I}{4\pi}\oint \frac{1}{\rad}\odif{\vbs{\ell}'}\\
    &=\frac{\mu_0I}{4\pi} \sum_{n=0}^{\infty}\frac{1}{r^{n+1}}\oint\ab(r')^nP_n(\cos\alpha)\odif{\vbs{\ell}'}
\end{align*}
ก็จะได้พจน์
\[
\vb{A}_\txt{mon}(\vb{r}) = \frac{\mu_0I}{4\pi r}\oint\odif{\vbs{\ell}'} = \vb{0}
\]
ตามที่คาด (เพราะจาก (\ref{gaussb}) เราสมมติไม่มี magnetic monopole)

ก่อนที่จะไปดูพจน์ dipole เราจะต้องพิสูจน์เอกลักษณ์หนึ่งที่จะต้องใช้ก่อน:
\begin{eqbox}{Claim}
    \oint_{\del\sur}\ab(\vb{c}\cdot\vb{r})\odif{\vbs{\ell}} = \int_\sur\odif{\vb{a}}\times\vb{c} = \vb{a}\times\vb{c}\label{claimbdip}
\end{eqbox}
\begin{proof}
    พิจารณา Stokes' Theorem บน $\vb{c}(\vb{c}\cdot\vb{r})$:
    \[
    \oint_{\del\sur}\vb{c}\ab(\vb{c}\cdot\vb{r})\cdot\odif{\vbs{\ell}} = \int_\sur\ab(\gd\times\vb{c}\ab(\vb{c}\cdot\vb{r}))\odif{\vb{a}}
    \]
    จากนั้นสลับการคูณของสเกลาร์ในฝั่งซ้ายและใช้กฎการคูณในฝั่งขวา จะได้
    \begin{align*}
        \oint_{\del\sur}\vb{c}\cdot\ab(\vb{c}\cdot\vb{r})\odif{\vbs{\ell}} &= \cancel{\int_\sur\ab(\vb{c}\cdot\vb{r})\ab(\gd\times\vb{c})\cdot\odif{\vb{a}}} - \int_\sur\ab(\vb{c}\times\gd\ab(\vb{c}\cdot\vb{r}))\cdot\odif{A}\\
        &= -\int_\sur\ab(\vb{c}\times\vb{c})\cdot\odif{a}\\
        &= -\int_\sur\vb{c}\cdot\ab(\vb{c}\times\odif{\vb{a}})
    \end{align*}
    ตัด $\vb{c}$ ทั้งสองฝั่ง ก็จะได้
    \[
    \oint_{\del\sur}\ab(\vb{c}\cdot\vb{r})\odif{\vbs{\ell}} = \int_\sur\odif{\vb{a}}\times\vb{c}
    \]
    ตามต้องการ
\end{proof}
ต่อมาเรามาพิจารณาพจน์ dipole:
\begin{align*}
    \vb{A}_\txt{dip}(\vb{r}) &= \frac{\mu_0I}{4\pi r^2}\oint r'\cos\alpha\odif{\vbs{\ell}'}\\
    &= \frac{\mu_0I}{4\pi r^2}\oint\ab(\vu{r}\cdot\vb{r}')\odif{\vbs{\ell}'}
\end{align*}
แต่จาก claim (\ref{claimbdip}) ก็จะได้ว่า $\oint_{\del\sur}\ab(\vu{r}\cdot\vb{r}')\odif{\vbs{\ell}'} = \int_\sur\odif{\vb{a}}\times\vu{r} = \vb{a}\times\vu{r}$ ดังนั้น
\[
\vb{A}_\txt{dip}(\vb{r}) = \frac{\mu_0I}{4\pi}\frac{\vb{a}\times\vu{r}}{r^2}
\]
เราจึงนิยาม \emph{magnetic dipole moment} ($\vb{m}$):
\begin{defbox}{ Magnetic Dipole Moment}
    \begin{equation}
        \vb{m}\equiv I\int\odif{\vb{a}} = I\vb{a}
    \end{equation}
\end{defbox}
ก็จะได้ว่า
\begin{eqbox}{พจน์ Dipole}
    \vb{A}_\txt{dip}(\vb{r}) = \frac{\mu_0}{4\pi}\frac{\vb{m}\times\vu{r}}{r^2}
\end{eqbox}
โดยสังเกตว่า $\vb{m}$ นี้ไม่ขึ้นกับจุดกำเนิด

\subsection{Dipole บริสุทธิ์}
เช่นเดียวกับ electric dipole บริสุทธิ์ เราสามารถสร้างจุดที่เป็น magnetic dipole บริสุทธิ์ได้ถ้ามองในลิมิต $I\to\infty$ และ $a\to 0$ โดยให้ $\vb{m} = I\vb{a}$ คงที่

สมมติให้ $\vb{m}$ ของ dipole บริสุทธิ์นี้ชี้ในทิศ $+z$ จะได้ว่า
\[
\vb{A}_\txt{dip}(\vb{r}) = \frac{\mu_0}{4\pi}\frac{m\sin\theta}{r^2}\vus{\upphi}
\]
ดังนั้นจึงได้สนามแม่เหล็กของ dipole บริสุทธิ์:
\[
\vb{B}_\txt{dip}(\vb{r}) = \gd\times\vb{A} = \gd\times\ab(\frac{\mu_0}{4\pi}\frac{m\sin\theta}{r^2}\vus{\upphi})
\]
เมื่อคำนวณออกมาจะได้ว่า
\begin{eqbox}{สนามแม่เหล็กของ Dipole บริสุทธิ์ในพิกัดทรงกลม}
    \vb{B}_\txt{dip}(r,\theta)=\frac{\mu_0}{4\pi}\frac{m}{r^3}\ab(2\cos\theta\,\vu{r}+\sin\theta\,\vus{\uptheta})
\end{eqbox}
ได้เหมือนกับสนามของ electric dipole บริสุทธิ์พอดี เราจึงหาสูตรในรูปทั่วไปที่ไม่ขึ้นกับพิกัดทรงกลมได้ในแบบเดียวกับ (\ref{dipe}) ก็จะได้ว่า
\begin{eqbox}{สนามแม่เหล็กของ Dipole บริสุทธิ์}
    \vb{B}_\txt{dip} = \frac{\mu_0}{4\pi}\frac{1}{r^3}\ab\Big(3\,\ab(\vb{m}\cdot\vu{r})\,\vu{r} - \vb{m})
\end{eqbox}