\chapter{แม่เหล็กสถิต}
\section{กฎแรง Lorentz}
\subsection{แรงแม่เหล็ก}

\begin{lawbox}{แรง Lorentz}
    ประจุ $Q$ ที่เคลื่อนที่ด้วยความเร็ว $\vv$ ในสนามแม่เหล็ก $\vB$ จะถูกแรงแม่เหล็กกระทำดังนี้:
    \begin{equation}
        \vF_\txt{mag} = Q\ab(\vv\times\vB)
    \end{equation}
    โดยถ้ามีทั้งสนามไฟฟ้าและแม่เหล็ก:
    \begin{equation}
        \vF = Q\ab\big(\vE + \ab(\vv\times\vB))
    \end{equation}
\end{lawbox}
การเคลื่อนที่ใน $\vB$ สม่ำเสมอที่น่าสนใจมีดังนี้:
\begin{enumerate}
    \item ถ้าประจุ $Q$ เคลื่อนที่ด้วยความเร็ว $\vv$ ในสนาม $\vB$ เพียงอย่างเดียว ส่วนของ $\vv_\perp$ จะทำให้เกิดการเคลื่อนที่วงกลมตามสมการ
    \[QBR=mv=p\]
    เมื่อ $p$ คือโมเมนตัม และได้
    \[\omega = \frac{QB}{R}\]
    จะเรียกว่า\emph{ความถี่ cyclotron}
    \item ถ้าประจุ $Q$ เริ่มจากหยุดนิ่งในสนาม $\vE$ และ $\vB$ ที่ตั้งฉากกัน ถ้าแก้สมการมาจะได้ว่าประจุจะเคลื่อนที่เป็นรูป cycloid ที่มีรัศมี
    \[R=\frac{E}{\omega B}\]
    เมื่อ $\omega$ คือความถี่ cyclotron และศูนย์กลางวงกลมที่ทำให้เกิดรูป cycloid จะเคลื่อนที่ด้วยอัตราเร็ว
    \[u=\omega R=\frac{E}{B}\]
\end{enumerate}
ต่อมาพิจารณางานจากแรงแม่เหล็ก:
\[\odif{W}_\txt{mag}=\vF_\txt{mag}\cdot\odif{\vl}=Q(\vv\times\vB)\cdot\vv\odif{t}=0\]
ดังนั้นได้ว่า
\begin{corbox}{งานของแรงแม่เหล็ก}
    แรงแม่เหล็กไม่ทำงาน:
    \begin{equation}
        W_\txt{mag}=0
    \end{equation}
\end{corbox}

\subsection{กระแสไฟฟ้า}
\begin{defbox}{กระแสไฟฟ้า}
    กระแสไฟฟ้า ($\vI$) ของจุดหนึ่งในสายไฟคือปริมาณประจุที่เคลื่อนที่ผ่านจุด ๆ นั้นต่อหน่วยเวลา หรือก็คือ
    \begin{equation}
        \vI = \lambda\vv
    \end{equation}
\end{defbox}
พิจารณา
\[\vF_\txt{mag}=\int\odif{\vF_\txt{mag}}=\int\ab(\vv\times\vB)\odif{q}\]
ดังนั้นในสายไฟจะได้
\begin{eqbox}{แรงแม่เหล็กบนสายไฟ}
    \vF_\txt{mag}=\int\ab(\vI\times\vB)\odif{\ell}\label{magforcei}
\end{eqbox}
หรือก็คือ
\begin{equation}
    \vF_\txt{mag}=\int\ab(I\odif{\vl}\times\vB)
\end{equation}

ต่อมา หากประจุที่เคลื่อนที่เป็นประจุจากความหนาแน่นในสองหรือสามมิติ เราจะนิยาม:
\begin{defbox}{ความหนาแน่นกระแสไฟฟ้า}
    สำหรับประจุที่ไหลบนผิวในสองมิติ ถ้าในแถบเล็ก ๆ ที่ขนานกับทิศในการไหลของกระแส $\odif{\vI}$ กว้าง $\odif{\ell_\perp}$ เราจะนิยาม\emph{ความหนาแน่นกระแสไฟฟ้าเชิงพื้นที่} ($\vK$) ว่า
    \begin{equation}
        \vK\equiv\frac{\odif{\vI}}{\odif{\ell_\perp}}=\sigma\vv
    \end{equation}
    สำหรับประจุที่ไหลในปริมาตรสามมิติ ถ้าในท่อเล็ก ๆ ที่ขนานกับทิศในการไหลของกระแส $\odif{\vI}$ มีพื้นที่ $\odif{a_\perp}$ เราจะนิยาม\emph{ความหนาแน่นกระแสไฟฟ้าเชิงปริมาตร} ($\vJ$) ว่า
    \begin{equation}
        \vJ\equiv\frac{\odif{\vI}}{\odif{a_\perp}}=\rho\vv
    \end{equation}
\end{defbox}
และเช่นเดียวกับ (\ref{magforcei}) จะได้ว่า
\begin{corbox}{แรงแม่เหล็กบนกระแสไฟฟ้าในสองและสามมิติ}
    \begin{equation}
        \vF_\txt{mag}=\int\ab(\vK\times\vB)\odif{a}
    \end{equation}
    และ
    \begin{equation}
        \vF_\txt{mag}=\int\ab(\vJ\times\vB)\odif{\tau}
    \end{equation}
\end{corbox}
